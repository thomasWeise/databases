\chapter*{Preface}%
\pdfbookmark{Preface}{preface}%
%
The goal of the course and book is to teach undergraduate and graduate students the topic of \glsreset{db}\pglspl{db}.
Our focus is a practice-oriented approach to the topic.
This means that each concept that we introduce or discuss is always accompanied by a rich set of examples.
In the course we will use many tools, ranging from%
%
\begin{itemize}%
\item the \postgresql\ \pgls{dbms}~\cite{TA2024DDAMWPAM,FP2023LP,OH2017PUAR,B2024PELUYDW}, over%
\item \yEd, a graph editor that can be used for conceptual modeling~\cite{SG2015MDAWY,Y2011YGEM},%
\item \libreofficeBase, which can be used as convenient front end for creating forms and reports for data in a \pgls{db}~\cite{FNFHWSKLSSGLFRSRPLJG2022BG7R1BOL7C,S2022L7PFEUU},%
\item \python~\cite{programmingWithPython}, a programming language which can connect to \postgresql\ using the \psycopg\ module~\cite{VDGE2010P}, to the%
\item \pgmodeler, a tool with which we can conveniently design logical \postgresql\ \db\ schemas~\cite{AES2006PPDM}.%
\end{itemize}%
%
After completing the course, you should be able to productively work with databases, at least at a beginner level.
You should be able to realize simple database-based applications.
And you should be able to navigate the huge ecosystem of different database management systems, tools, and paradigms in this field in order to pick the right tool for the right problem.

This book is intended to be read on an electronic device.
Please do not print it.
Help preserving the environment.

This book is work in progress.
It will take years to be completed and I plan to keep improving and extending it for quite some time.

This book is freely available.
You can download its newest version from \expandafter\url{\databasesUrl}.
This version may change since this book is, well, work in progress.

The book consists of two types of material:
Materials that the author~(Thomas Weise) has created by himself and such that have been created by others.
The vast majority of the material is teaching material created by the author.
This and only this material is released under the \emph{Attribution-NonCommercial-ShareAlike~4.0 International license}~(\href{http://creativecommons.org/licenses/by-nc-sa/4.0}{\mbox{CC~BY-NC-SA~4.0}}).
However, the book also includes some images and figures created by others, which are marked explicitly and licensed under their authors' terms.
For example, all logos and trademarks are under the copyright of their respective owners.

You can cite this book~\cite{databases}, e.g., by using the following Bib\TeX:%
%
\begin{lstlisting}[style=text_style]
@book{databases,
 author = {Thomas Weise},
 title = {Databases},
 year = {2025},
 publisher = {Institute of Applied Optimization,
              School of Artificial Intelligence and Big Data,
              Hefei University},
 address = {Hefei, Anhui, China},
 url = {https://thomasweise.github.io/databases}
}
\end{lstlisting}
%
%
\begin{sloppypar}%
The text of the book itself is also available in the repository \url{\databasesRepo}.
There, you can also submit \href{\databasesRepo/issues}{issues}, such as change requests, suggestions, errors, typos, or you can inform me that something is unclear, so that I can improve the book.
Such feedback is most welcome.
The book is written using \LaTeX\ and this repository contains all the scripts, styles, graphics, and source files of the book~(except the source files of the example programs).%
\end{sloppypar}%
%
\strut\vfill\strut%
%
\copyrightBlock{2025}%
%
\clearpage%
%
\strut\vfill\strut%
%
\begin{center}%
\noindent\resizebox{\linewidth}{!}{%
\begin{tabular}{c@{~~~~~~~~}c}%
\includegraphics[width=7cm]{\currentDir/bookUrl}&\includegraphics[width=7cm]{\currentDir/courseUrl}\\\relax%
{\huge{[\expandafter\href{\databasesUrl/databases.pdf}{book pdf}]}}&{\huge{[\expandafter\href{\databasesUrl}{course website}]}}\\%
\end{tabular}}%
\\[12pt]\noindent%
\resizebox{0.85\linewidth}{!}{\expandafter\url{\databasesUrl}}%
\end{center}%
%
\strut\vfill\strut%
This book was built using the following software:%
\gitOutputWithStyle{}{}{text/front/preface/dependencyVersions.sh}{versions}{}{text_style}%
