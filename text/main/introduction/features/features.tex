\hsection{Features that we want from a Database}%
%
So far, we developed some rough ideas about what a \pgls{db} is and what kind of features it offers us.
Let us try to come up with a clearer list.

We will work through this list by imagining that we are building a database for a bank.
The database is supposed to store customers, accounts, transactions, and bank employees.
Without getting too specific, we can iterate our expectations based on this idea.%
%
\hsection{Data Modelling and Representation}%
First of all, the \pgls{dbms} must give us the tools into our hands to create and implement a model for our data.
Matter of fact, we can distinguish several facets of modeling and representing data.
We here focus on the \emph{relational} data model, based on tables on the relations between them.
There can also be \pglspl{db} using other models, such as graphs, hierarchies, objects, and such and such.

Based on the data model, we must be able to define a conceptual or logical schema for our \pgls{db}.
For example, we want to specify which kind of tables exist, e.g., a table with customer information, a table with bank employee information, a table for the accounts that exist, and a table with all the transactions.
We can define which information we want to store about customers, say their name, ID number, mobile phone number, etc.
This includes some basic validity constraints, e.g., names cannot be empty, ID numbers must follow the standard Chinese ID number scheme, and so on.
We must also be able to specify the relationships between these tables.
For example, each bank account must be associated with exactly one customer, but one customer may have multiple accounts.
Ideally, this can be done using a simple programming language.

This logical schema can be separate from the physical schema of the \pgls{db}, namely how the data is actually organized on the disk.
The \pglspl{dbms} should allow the \pgls{db} designers to, for instance, define indexes for tables to speed up the access.

Nevertheless, the concept of \emph{Data Independence} is important:
There are different levels of abstraction.
The application programs that access the data based on the logical schema should be independent from the physical organization of the data.
It should not really matter to them whether records are stored in a hash, a B-tree, or as simple sequences in a file.
If we create an easy-to-use application as interface for our banking \pgls{db}, this interface would interact with the logical view of the data via the \pgls{dbms}.
How the data is actually stored does not matter to our application.
This is called \emph{physical data independence}.

Let's say we create two applications:
We create an application for the bank employees to manage accounts and customers.
We also create a web application through which the customers can log into their bank accounts and make transfers.
The first application may offer much more information to the bank employees.
Customers have do not have access to the accounts of other customers, for example.
The web application for customers can also not access information about the bank employee who is managing their accounts.
Therefore, it should work independently from such information.
If the structure of these information (that it cannot see) is changed later, then this should have no impact on that program.
This is called \emph{logical data independence}.
\endhsection%
%
\hsection{Data Availability} %
The \pgls{dbms} makes a collection of data is available to users and applications in a meaningful format at a reasonable performance.
We expect that the speed of reading, writing, changing, deleting, and adding of data to the \pgls{db} be high.
A simple and reasonably abstract programming language should be provided for querying and updating the \pgls{db}.
There should also be tools available that further simplify the modeling, editing, and sorting of data as well as the generation of reports from data.%
\endhsection%
%
\hsection{Data Integrity}%
The correctness and validity of the data is ensured.
This includes several aspects:
First, the modeled relationships must be enforced.
If it is stated each bank account is associated with exactly one customer, then there must never be a bank account record that is not associated with an existing customer.
Also, there can never be two bank accounts with the same account~ID.
This property is called \emph{Consistency}.

Let's say that for each account, we store the current balance.
If a customer deposits money, then we can simply add the amount to their account's balance.
This is one operation where not much can ge wrong.
However, if money is transferred from one account of our bank to another account, then this \inQuotes{looks like} two operations.
But it actually is a single \emph{transaction}.
A \pgls{db} should be consistent before and after each transaction.

\emph{Atomicity} means that such a transaction will either complete successfully in its entirety or fail.
If the transaction is completed successfully, i.e., \emph{commits}, then all the changes become visible.
If the transaction is aborted, no change is performed.
No transaction can complete partially, it is all-or-nothing.
This means that either the money is correctly subtracted from one account and added to another \emph{or} nothing happens.
It must never happen that money is subtracted from one account but does not arrive in another due to an intermediate error.%
\endhsection%
%
\hsection{Concurrency Support and Isolation}%
Closely related to data integrity is the support for concurrency.
Multiple users and multiple processes can access and modify the \pgls{db} at the same time.
The \pgls{dbms} preserves the integrity of the \pgls{db} and of all the views that the users and process have on the data.

Different concurrent updates to the same account must take place atomically and in isolation.
This means that if money is transferred from account~$A$ to account~$B$ while simultaneously money is transferred from account~$B$ to account~$C$, the two transactions should not influence each other.
Both transactions should be isolated.%
\endhsection%
%
%
\hsection{Durability and Data Safety}%
A basic feature that any \pgls{db} needs to support is \emph{Durability}:
Once a transaction is committed, its changes are permanent.
The changes caused by the transaction are saved to the database permanently.

The \pgls{dbms} has to ensure that data can be preserved in case of unforeseen situations.
Indeed, durability extends also to system crashes and failures.
This includes the support of checkpoints and recovery after restarts.

Of course, there are limits to this:
At some lower layers, \pglspl{db} are stored as files in the filesystem.
If the underlying hard disk fails, these files can be destroyed.
There is no magical way a \pgls{dbms} can guard against this.
No system can be entirely safe from hard disk failures and crashes.

Therefore, \glspl{dbms} need to offer methods to back up the data, i.e., to create copies of the \pglspl{db} and store them on other devices.
Then, as long as reasonable backup strategies are employed, it must be possible to re-create the data after the original \pgls{db} was lost.%
\endhsection%

The features \emph{A}tomicity, \emph{C}onsistency, \emph{I}solation, and \emph{D}urability together are often called \emph{ACID}~properties.
%
\hsection{Data Privacy and Security}
The \pgls{dbms} must enable the protection of data against access from unauthorized access from both within and outside of the organization.

It must be possible to specify roles for users and processes that define which data they can access or modify.
For example, the \pgls{dba} of the bank's \pgls{db} needs to be able to modify the structure of the \pgls{db}, maybe add and modify tables.
A normal bank employee should not be able to do that.
But they may be allowed to enter new records into certain tables of the the \pgls{db}, for instance, to add customers or to create new accounts.
Thinking this through even further, a bank employee may be assigned as manager to some customer's accounts.
The employees should only be able to see the information in these accounts.
All of this can be done by assigning roles to users and processes and access rights to roles.

Another aspect of this is the so-called Three-Schema Architecture:
It should be possible to provide different views on the data for different users and applications.
As already mentioned before, the application for bank employees should be able to \inQuotes{see} other data than an application for customers.
In other words, at the bottom layer of the \pgls{db} is the physical schema managing how data is stored.
On top of that sits the conceptual (or logical) layer, where the structure of the tables and the relationships between them are defined.
On the highest level, different views on the data can be provided for different applications.
This way, users can only see the data that they need to be able to see and nothing more.

This goes hand-in-hand with security measures.
Examples could be that all access to the \pgls{db} is password protected.
Communication with the \pgls{db} happens over encrypted connections.

A \pgls{dbms} should also support accounting and an audit trail.
It must be possible to log information regarding which user changed which dataset and when.
This is required to pass certain government, financial, or ISO certification audits.
\endhsection%
%
\hsection{Summary}%
It is quite obvious that the basic filesystem only offers very few of the properties discussed above.
Using simple data formats like \pgls{CSV} or even Excel tables will not be a good choice either.
While one might realize concurrency by placing multiple interrelated such documents on a shared folder, there certainly would not be any \emph{ACID} guarantees.
Multiple people editing them at the same time will wreak havoc.
Structured formats like \pgls{json}, \pgls{xml}, or \pgls{yaml} would allow us to represent complex data in single documents, but this would make the multi-user access situation even worse.
And none of these solutions could provide any means to protect data integrity, let alone access control.

Instead, we require a software layer between the \pgls{db} and the user processes.
Honestly, we do not even really care of this software layer actually stores the data, how the data is organized internally, or how \emph{ACID} is implemented.
As long as this software, the \pgls{dbms}, offers us the above features, we will be happy.%
\endhsection%
\endhsection%
%
