\hsection{Software}%
\label{sec:software}%
%
A wide range of \pglspl{dbms} for \pglspl{rdb} exist.
A few we already mentioned before.%
%
\hsection{Open Source Relational Database Management Systems}%
A very big chunk of the \pglspl{rdb} in many scenarios is managed by open source \pglspl{dbms}.
These systems are available for free and their source code is readily available in the internet.
Large communities exist around them that can offer all kinds of advise.
This makes them very attractive for developers and users.
Several of these systems are around since the 1990s, so they are mature and tested products~\cite{C20245YOQ}.

\mysql\ is a \pgls{DBMS} for \pglspl {rdb}~\cite{WAM2002MRMDFTS,TA2024DDAMWPAM,BT2021HPM,RGS2021BTOTONAMDFPC,D2015LMAM}.
It was originally developed by Michael Widenius and David Axmark at the Swedish company \mysql~AB and released in 1996~\cite{C20245YOQ}.
It soon gained widespread use as part of the \lampStack, i.e., a system setup for web applications based on the \linux\ operating system, the Apache webserver, the \mysql\ database, and the server-side scripting language~PHP~\cite{C2022HAFTLS,H2020ULU2E}.
\mysql~AB was acquired by Sun Microsystems in 2008, which in turn was acquired by Oracle in 2010~\cite{C20245YOQ}.

After the acquisition by Oracle, some of the original developers of \mysql, including Michael Widenius, created a fork of \mysql:~\mariadb~\cite{R2014MM,B2019LTMEELFFSAA,D2015LMAM,AA2018QAWMV1ITSQ,AA2018QAWMV2IDQ}.
They promise that \mariadb\ will stay open source forever.

\postgresql~\cite{TA2024DDAMWPAM,FP2023LP,OH2017PUAR,B2024PELUYDW} is an object-relational \pgls{dbms}, meaning that it supports concepts from \pgls{OOP}, such as inheritance relationships between tables.
It also supports many additional types like \pgls{JSON} objects and geometric types.
\postgresql\ emerged from the POSTGRES project, the successor of the INGRES project at the University of Berkeley in California~\cite{C20245YOQ}.
It may be the most fully-featured of the existing open source \pgls{SQL} \pglspl{dbs}

The \pgls{SQL} \pgls{db} with the widest distribution is \sqlite~\cite{WB2019RHSOOS,GPBHKP2022SPPAF,C20245YOQ}.
It breaks with the common approach to offer access to the \pgls{db} in a client-server scheme.
Instead, it is directly loaded as a library in the process that uses.
Today, it is installed on nearly every smartphone, computer, web browser, television, and automobile~\cite{WB2019RHSOOS,GPBHKP2022SPPAF,C20245YOQ}.
It was first released in 2000 and its core designer Richard Hipp received the SIGMOD Systems Award in 2017~\cite{C20245YOQ}.

The concept of \pglspl{dbms} that can work on stand-alone files is also implemented in the open source office suite \libreoffice~\cite{DF2024LTDF,GL2012LTSOOSSCBAFACSOL,S2022L7PFEUU}: \libreofficeBase~\cite{FNFHWSKLSSGLFRSRPLJG2022BG7R1BOL7C,S2022L7PFEUU}.
Base is more than a \pgls{DBMS} working on single file.
Instead, it offers a powerful user interface that can also connect to databases such as \mysql, \mariadb\ and \postgresql.
The user interface allows you to conveniently design tables, relations, forms, and reports for the \pglspl{db} it connects to.
\pgls{libreofficeBase} is a free alternative to the commercial product Microsoft Access~\cite{SSI2023MA2BTA,B2020HOMA2,UC2021AFD}, which offers a similar functionality.

Many of the examples used in this book will be implemented based on \postgresql\ and we will also play around with \libreofficeBase.%
%
\endhsection
%
\endhsection%
%
