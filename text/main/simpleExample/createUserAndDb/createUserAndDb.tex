%
\hsection{Creating a User and the Database}%
\label{sec:simpleExample:createUser}%
%
Our boss has asked us to create an application for managing products, customers, and demands.
Since these are different types of entities with different properties, managing with an \microsoftExcel\ or \libreofficeCalc\ spreadsheet makes little sense.
We will need a \db.

The first step to fulfill this request would thus be to create a new and empty \db.
We already installed \postgresql~(see \cref{sec:installingPostgres}).
It is running on a dedicated \server\ computer in our small \pgls{IT} department / office.
This \db\ \server\ will host all the \pglspl{db} of the company, probably ranging from payroll data to fancy business analytics.

However, when we discuss the idea for a new \db\ application with our boss, they state that they want to have full access to the new \db.
Of course, they are not a trained \glsreset{dba}\pglspl{dba}.
Many things could go wrong if we would design a nice \db\ and then unleash untrained personnel onto it.
We would be even more reluctant to give them administrative access to the complete \dbms\ \server.
This \server\ could house many different \pglspl{db} for different purposes.
We want to keep it under our control and, at least, limit the \inQuotes{full access} of our boss to only this one single new \db.

In a first step, we would therefore create a new role or user account on our \dbms.
This account should only be able to access the new factory \db.
If they make a mistake, this mistake will only affect this single \db.
If some outside attacker can obtain their password, then the impact will only be limited to this \db\ and not affect, e.g., payroll data or other confidential data in other \pglspl{db}.
After such a user account is created, we can then create the actual \db\ and have the new user be its owner.%
%
\hinput{user}{user.tex}%
\hinput{db}{db.tex}%
%
\FloatBarrier%
\endhsection%
%
