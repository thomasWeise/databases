%
\hsection{Cleanup After the Example}%
\label{sec:simpleExampleCleanup}%
%
\gitLoadAndExecSQL{factory:cleanup_inside_database}{}{factory}{cleanup_inside_database.sql}{factory}{boss}{superboss123}%
\listingSQLandOutput{factory:cleanup_inside_database}{%
User \textil{boss} deletes all tables and views inside the database, in the inverse order of their creation.%
}{}%
%
\gitLoadAndExecSQL{factory:cleanup_database_and_user}{}{factory}{cleanup_database_and_user.sql}{}{}{}%
\listingSQLandOutput{factory:cleanup_database_and_user}{%
Delete the database \textil{factory} and the user \textil{boss}. %
This must be executed by the administrator account \textil{postgres}.%
}{}%
%
We are now approaching the end of this brief journey crisscrossing the domain of \pglspl{rdb} to get a rough impression what they are.
To finish it, let us delete all the things we created.
Of course, we have to do that in the opposite order in which we created them.

In \cref{lst:factory:cleanup_database_and_user}, we first delete the view~\sqlil{sale}.
We can do this by the command \sqlil{DROP VIEW sale;}\sqlIdx{DROP!VIEW}~\cite{PGDG:PD:DV}.
The \sqlil{DROP} is the \sql\ command for deleting things.
The \sqlil{VIEW} is specified to make clear what type of object we want to delete, to prevent us from accidentally deleting something else.
Then we give the name of the view to delete, in our case, that is \sqlil{sale}.
However, we make this command a bit more handy:
We insert an \sqlilIdx{IF EXISTS} inbetween the object type and the view's name.
This condition is self-explanatory:
If a view of the provided name exists, then it is deleted.
If not, then nothing happens.
Without the \sqlilIdx{IF EXISTS}, this would cause an error.

Well, we do know that the view exists, so why are we mentioning this condition here?
Because it has some interesting use cases.
If we wanted to create a backup copy of our entire \db, then one thing we could do is to export the whole \db\ as a single large \sql~script~\cite{PGDG:PD:SD}.
This script could, basically, be the concatenation of all of our listings.
Running this script would re-create the \db\ in exactly its present state.

Except that it won't.
If the \db\ already exists, it will fail.
Or maybe if we had a crash or a user issued some faulty \sql\ commands wreaking havoc, then the \db\ maybe now exists only partially.
Maybe some tables or views have disappeared, maybe some rows in some tables are missing.
One method to make the \inQuotes{backup \sql~script} robust to deal with such issues is to add \sqlil{DROP ... IF EXISTS} clauses \emph{before} the commands for re-creating each table or view.
Then we can even restore the \db\ while the \db\ still exists.%
%
\begin{sloppypar}%
Anyway, we use the same method to delete the three tables.
We issue the commands \sqlil{DROP TABLE IF EXISTS demand;}\sqlIdx{DROP!TABLE}\sqlIdx{IF EXISTS}, \sqlil{DROP TABLE IF EXISTS product;}, and \sqlil{DROP TABLE IF EXISTS customer;}\sqlIdx{DROP!TABLE}\sqlIdx{IF EXISTS}~\cite{PGDG:PD:DT}.
Notice that we delete \sqlil{demand} before we delete \sqlil{customer} and \sqlil{product} because of the \sqlilIdx{REFERENCES} constraints.%
\end{sloppypar}%
%
All of these deletion steps are done by using the user \textil{boss} and their password \textil{superboss123}.
Let us finally also get rid of the entire \db\ and of that user as well in \cref{lst:factory:cleanup_database_and_user}.
First, we delete the \db\ by executing \sqlil{DROP DATABASE IF EXISTS factory;}\sqlIdx{DROP!DATABASE}\sqlIdx{IF EXISTS}~\cite{PGDG:PD:DD}.
Then we remove the user via \sqlil{DROP USER IF EXISTS boss;}\sqlIdx{DROP!USER}\sqlIdx{IF EXISTS}~\cite{PGDG:PD:DU}.
Notice that we log in as the \pgls{dba} user \textil{postgres} to do that.%
%
\FloatBarrier%
\endhsection%
%
