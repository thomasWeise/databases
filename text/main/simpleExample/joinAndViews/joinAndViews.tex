%
\hsection{Join-based Select and Views}%
%
So far, things are going well.
On the positive side, we have learned that we can divide the data of factory into separate aspects and then store it into different tables.
This is nice.
We also learned that we can ensure the integrity of the data inside a single table in various ways.
For example, we can use datatypes that strictly enforce their domains.
\sqlilIdx{DECIMAL}, for example, makes sure that numbers are represented exactly without loss of precision to a fixed number of decimals~\cite{PGDG:PD:NT}.
\sqlilIdx{DATE} ensures that valid dates based on the Gregorian calendar are entered~\cite{PGDG:PD:HU,G1582IG}.
We can also define constraints, ranging from ensuring that all columns are entered~(\sqlilIdx{NOT NULL}) over preventing name clashes~(\sqlilIdx{UNIQUE}) to sanity checks~(via \sqlilIdx{CHECK} \sqlIdx{CONSTRAINT}constraints)~\cite{PGDG:PD:C}.
We also learned that we can even ensure the consistency of the data in our \db\ across different tables via the \sqlilIdx{REFERENCES} constraint~\cite{PGDG:PD:FK}.
So we cannot just have tables, they can have strongly-typed columns and the data integrity can be enforced throughout the complete \db.
This puts \pglspl{rdb} well ahead the spread sheets produced by the likes of \pgls{excel} or \pgls{libreofficeCalc}.

However, on the negative side, we have to admit that using \dbs\ is more complicated.
And we are certainly losing in terms of readability of our data.
Indeed, the meaning of the \sqlil{demand} data shown in \cref{exec:factory:insert_into_table_demand} is not obvious without knowing the contents of the other tables.
Now we will see that we can also \emph{use} these contents of other tables directly in our queries, to produce clear and human-readable output.%
%
\hsection{Joining Tables}%
\gitSQLAndOutput{\databasesCodeRepo}{factory}{select_customer_demand.sql}{factory}{boss}{superboss123}{postgres.sh}{factory:select_customer_demand}{%
Get the per-customer demands.%
}%
%
We now want to have a list of all of our customers, and for each customer we want to see the \sqlil{id} values of all of the demands (orders) they have issued.
We now step-by-step build the query used in \cref{lst:factory:select_customer_demand}.
So the result of our query should have two columns.
The first column should hold the customer names.
Let's call it \sqlil{customer}.
The second column should hold the \sqlil{id} of the order the customer made.
Let's call this column \sqlil{demand_id}.
Now if a customer issued multiple orders, then they should appear multiple times in this table, once for each demand.
If a customer did not issue any orders, then they should still appear, but only once and with a \sqlilIdx{NULL} value in the \sqlil{demand_id} column.

This means that first, we need all the customer names.
A \sqlil{SELECT name FROM CUSTOMER;} would do this for us.
Now we need to cross-reference the customer \sqlil{id} values in table \sqlil{customer} with the column \sqlil{customer} in table \sqlil{demand}.
We also want to list the customers who did not issue any order.%
%
\begin{sloppypar}%
What we want to do is to apply a so-called \sqlil{LEFT JOIN}\sqlIdx{JOIN!LEFT}, also called \sqlIdx{LEFT OUTER JOIN}\sqlIdx{JOIN!LEFT OUTER}, of the table \sqlil{customer} to the table \sqlil{demand}~\cite{PGDG:PD:JT}.
The syntax of a left join is basically \sqlil{SELECT <something> FROM table_1 LEFT JOIN table_2 ON (table_1.a = table_2.b)}.
Here \sqlil{<something>} can be any column from \sqlil{table_1} or \sqlil{table_2}.
The columns from \sqlil{table_1} that we want need to be prefixed by \sqlil{table_1.}.
The columns from \sqlil{table_2} that we want need to be prefixed by \sqlil{table_2.}.
The important part of the \sqlil{LEFT JOIN}\sqlIdx{JOIN!LEFT} is the \sqlilIdx{ON} condition.
It can be an arbitrarily complex condition, involving \sqlilIdx{AND} and \sqlil{OR} and whatnot.
However, in the simplest form, it just picks a column from \sqlil{table_1}, here \sqlil{table_1.a} and says that its value must be the same as a column in \sqlil{table_2}, here \sqlil{table_2.b}.%
\end{sloppypar}%
%
We apply this to our situation.
\sqlil{table_1} is obviously \sqlil{customer} and \sqlil{table_2} is \sqlil{demand}.
The thing that we want to select is \sqlil{customer.name} and the corresponding values of \sqlil{demand.id}.
The fields that need to match are \sqlil{customer.id} and \sqlil{demand.customer}.%
%
\begin{sloppypar}%
In \cref{lst:factory:select_customer_demand}, we \sqlil{SELECT customer.name, demand.id} and we do so \sqlil{FROM customer LEFT JOIN demand ON (customer.id = demand.customer)}.
When executed, this query goes through the table~\sqlil{customer}.
For each record, it will take the field~\sqlil{id}, i.e., \sqlil{customer.id}.
It will then search for any row in the table~\sqlil{demand} whose field \sqlil{customer}, i.e., \sqlil{demand.customer} has the same value.
For any such row, it will write a row to the output with the \sqlil{name} of the customer and the \sqlil{id} of the row in \sqlil{demand}.
If and only if no such row exists for a customer, i.e., if the customer did not yet make any purchase, it writes a row to the output with the \sqlil{name} of the customer and \sqlilIdx{NULL} as the \sqlil{demand}~id.
This means that we get the demands associated with each customer name.
We also see which customer did not make any purchase.%
\end{sloppypar}%
%
To clean up the output, we rename \sqlil{customer.name AS customer_name}\sqlIdx{AS} and \sqlil{demand.id AS demand_id}\sqlIdx{AS} to make it clearer which name and id values will be displayed.
We also add an \sqlil{ORDER BY customer_name}\sqlIdx{ORDER BY} to sort the output by customer name.

The result of this query, shown in \cref{exec:factory:select_customer_demand}, has nine rows.
We have eight demands, which are associated with customers Bebba, Bebbo, and Bibbo.
Bobbo did not yet make any purchase, so he appears as a single row with \sqlilIdx{NULL} associated as \sqlil{demand_id}.
The \psql\ \pgls{client} just leaves \sqlil{demand_id} blank in the output.

We now have cross-referenced two tables!
Let us take this a step farther and ask \inQuotes{How many orders did each customer make so far?}
We again answer this question in \cref{lst:factory:select_customer_demand}.
Doing this is rather easy:
First, we replace \sqlil{demand.id AS demand_id} in the query with \sqlil{COUNT(demand.id) as demands}\sqlIdx{COUNT}.
We already learned that \sqlilIdx{COUNT} just counts rows over groups.
But in order to let it count meaningfully, we need to create these groups.
All we have to do for this is to add a \sqlil{GROUP BY customer_name}\sqlIdx{GROUP BY} to the query.
All rows that share a \sqlil{customer_name} go into the same group.
\sqlilIdx{COUNT} now counts the rows in these groups.
The output in \cref{exec:factory:select_customer_demand} shows us that Bebba made two orders, Bebbo~four, and Bibbo also two.
Bobbo so far has zero orders in his name.

With this, we now know how to cross-reference tables.
We have essentially solved the basic annoyance mentioned before:
Yes, our data is distributed over different tables and without knowing what product a product~id refers to and which customer corresponds to which customer~id, it is hard to understand anything.
However, we can just join tables pull all the information together in a query.%
%
\FloatBarrier%
\endhsection%
%
\hsection{Views as Virtual Tables}%
\gitSQLAndOutput{\databasesCodeRepo}{factory}{create_view_sale.sql}{factory}{boss}{superboss123}{postgres.sh}{factory:create_view_sale}{%
Creating a view, i.e., a stored \sql\ query that can be treated like a table, to list the demands in human-readable form.%
}%
%
We are now able to cross-reference different tables and to pull data together.
What we want to do is to create a query where, for each demand, we see the customer name, the product name, the product price, the ordered amount, and the order date.
The result of this query would then be a clear and human readable overview of the business transactions of our company.
Based on the previous example, where we joined the table \sqlil{customer} with the table \sqlil{demand}, we anticipate that we will need two \inQuotes{joins} now.
We will need to combine the \sqlil{demand} table with the \sqlil{customer} table and with the \sqlil{product} table.
The query will therefore probably be more complicated.

Moreover, it is likely that we will need this query structure several times.
For example, we probably want to be able to query all the orders that were made in a specific year.
This would basically extend the same query with a \sqlilIdx{WHERE} clause where we limit the demand dates.
Or maybe we want to sum up the cash flow per customer, to see who our most lucrative customers are and to maybe make them some special offers
Or maybe we want to summarize the total sales per product.
This would tell us which of our products sell well and which don't.

Every time we would re-write the query, just a bit differently.
Of course, this is not a good approach.
On one hand, every time we write a query, there is a chance that we introduce a mistake.
On the other hand, if we ever decided to change the structure of \db, maybe by allowing multiple items per demand, then we are stuck with a heap of queries that all need to be changed.

Luckily, the \sql\ language and \pglspl{rdb} offer us a tool for this situation:
We can store the query as so-valled \emph{view}~\cite{PGDG:PD:CV}.
A view is basically a stored query that we can work with exactly as if it was a table.
We can apply other \sqlil{SELECT}\sqlIdx{SELECT{\idxdots}FROM} queries on top of it.

Thus, we now will perform two steps:
First, we design the new query that joins three tables.
Second, we store this query as view so that we can re-use it whenever it pleases us.

Let us construct the new query.
We first define what we want in a way more close to \sql.
\inQuotes{For each row in \sqlil{demand}, we need the corresponding row in \sqlil{customer} and the corresponding row in \sqlil{product}.}
We know that there always exists exactly one \sqlil{customer} row, because we defined the foreign key \sqlilIdx{REFERENCE} constraint that enforces this.
We also know that there always exists exactly one \sqlil{product} row, because we defined the foreign key \sqlilIdx{REFERENCE} constraint that enforces this.
So it cannot be that we have a row \sqlil{demand} but no corresponding row in \sqlil{customer} or \sqlil{product}.
We already know that it is totally possible the other way around, i.e., that there is a row in \sqlil{customer} without any related row in \sqlil{demand} -- but this is not important here.

Given this information, we \emph{could} use the \sqlil{LEFT JOIN}\sqlIdx{JOIN!LEFT} from before.
It would generate rows also for rows of \sqlil{demand} that do not match to any \sqlil{customer} or \sqlil{product}.
But such rows do not exist anyway.
However, what we actually would like to use here is an \sqlil{INNER JOIN}\sqlIdx{JOIN!INNER}.
The \sqlil{INNER JOIN} only creates output rows when the rows of all involved tables match each other.%
%
\begin{sloppypar}%
But how do we go about joining \emph{three} tables?
Well, let's begin by doing basically the same thing as before:
We \sqlil{SELECT <something> FROM demand} and then \sqlil{INNER JOIN customer ON (customer.id = demand.customer)}.
For each row in \sqlil{demand}, this gives us the corresponding row in \sqlil{customer}.
It also would drop each row in \sqlil{demand} for which no row in \sqlil{customer} exists from the output (but that cannot happen anyway).
Now we simply \emph{chain} the next \sqlil{INNER JOIN} by writing it directly after that!
We write \sqlil{INNER JOIN product ON (product.id = demand.product)}.
This will also select the corresponding row from table \sqlil{product}.
If such a row does not exist, then the whole current combined row from \sqlil{product} and \sqlil{product} would be dropped from the output.
Thanks to our \sqlilIdx{REFERENCE} constraints in table \sqlil{demand}, this can never happen anyway.%
\end{sloppypar}%
%
\begin{sloppypar}%
We now have nicely pulled the customer and product data for each demand order.
We now choose the columns that we want to output and replace the \sqlil{<something>} we first nonchalantly wrote when beginning to design the query.
First, we want to get the customer name, so we write \sqlil{customer.name AS customer_name}\footnote{%
As we discussed before, customer names are not necessarily \sqlilIdx{UNIQUE} in our \db\ design. %
This would be a flaw of this query, as it does not distinguish two different customers with the same name. %
In real systems, you have customer numbers or unique usernames to avoid this issue. %
But this here is just a small example\dots}.
We also want to get the product name, so we write \sqlil{product.name AS product_name}.
Notice how both the table \sqlil{customer} and the table \sqlil{product} have a column called~\sqlil{name}.
However, there can never be any confusion here, because we refer to these columns as \sqlil{customer.name} and \sqlil{product.name}, respectively.
Next we want the list product price and amount associated with an order, i.e., \sqlil{product.price AS price} and \sqlil{demand.amount AS amount}.
Finally, we also return the order date and therefore add column \sqlil{demand.ordered as ordered}.%
\end{sloppypar}%
%
For good measures, we add an \sqlilIdx{ORDER BY} clause and sort the output by \sqlil{customer_name}.
If two entries have the same \sqlil{customer_name}, then we sort the one with the earlier order date (\sqlil{ordered}) first.
If two entries have the same \sqlil{customer_name} \emph{and} \sqlil{ordered} date, then we sort the one with the lexicographically smaller \sqlil{product_name} first.
We try to resolve further draws by \sqlil{price} and \sqlil{amount}.

The second thing we wanted to do was to save this query into the \db\ as a \emph{view}.
This way, we can re-use it whenever we want.
Storing it as a view is very very simple.
We first need to pick a name for it.
Let's call it \sqlil{sale}.
Then, we just need to write \sqlil{CREATE VIEW sale AS}\sqlIdx{CREATE!VIEW}\sqlIdx{AS} right before our new query.
If we do this, the query is not actually executed.
It is stored under the name \sqlil{sale} as a view.
We do this in \cref{lst:factory:create_view_sale}.

What if we want to actually execute the query?
Well, we treat \sqlil{sale} as if it was a table.
We write \sqlil{SELECT * FROM sale;}.
We do this as well in \cref{lst:factory:create_view_sale}.

The result is shown in \cref{exec:factory:create_view_sale}.
Eight beautiful human readable rows of data.%
%
\FloatBarrier%
\endhsection%
%
\hsection{Using our View}%
\gitSQLAndOutput{\databasesCodeRepo}{factory}{select_from_view_sale_1.sql}{factory}{boss}{superboss123}{postgres.sh}{factory:select_from_view_sale_1}{%
Compute the per-customer and per-product sales based on the view \textil{sale}.%
}%
\gitSQLAndOutput{\databasesCodeRepo}{factory}{select_from_view_sale_2.sql}{factory}{boss}{superboss123}{postgres.sh}{factory:select_from_view_sale_2}{%
Compute the per-customer and per-product sales based on the view \textil{sale}.%
}%
%
\FloatBarrier%
\endhsection%
%
