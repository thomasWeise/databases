Before we step-by-step learn about the features and intricacies of \pglspl{db}, let us look into a simple example as a teaser.
So far, you have learned about the history of databases.
Many courses that I know then, from here on, focus on several quite useful things:
First, an outline of the \db\ design process is given.
Second, you will learn about modeling data, how to draw \pglspl{ERD}.
Third, you will learn the so-called $\sigma$\nobreakdashes-algebra, which is an mathematical notation abstracting from the technological aspects of \pglspl{rdb}.
Fourth, they will tell you how to select, insert, and remove data using~\sql.
Some will give really practical examples and let you explore how to work with a \db\ via homeworks.
Fifth and finally, you are taught how \pglspl{db} work internally, what datastructures they use, and how they achieve efficient storage and high query performance.

I agree that it is a good didactic method to approach \pglspl{db} from several different angles as an abstract subject.
However, when I learned something as a student, I never really learned it this way.
I learned it by doing.
I am a believer that practical ability is nine tenth of mastering a field.
And practical ability comes easiest by playing around with the basic tools.

Also, as far as I can see, quite a few courses seem to treat \pglspl{db} as something \inQuotes{single,} something that exists \inQuotes{separately} from other things.
But this is not necessarily true.
Many of the subjects you learn elsewhere may be connected to \pglspl{db}.
For example, you often have programming classes, where you learn a programming language like \python~\cite{programmingWithPython}.
What is the connection here?
You will learn that, yeah, \pglspl{db} are often accessed from program code, maybe by so-called \emph{application servers}, that implement the business logic of an enterprise.
How does that work?
Often, you will not learn that.

Then again, \pglspl{db} are also often accessed from \pglspl{GUI}.
They maybe allow us to enter information via convenient forms.
Because users like office workers would feel puzzled if asked to enter and retrieve information using \sql\ queries.
Maybe we can also print reports with information extracted from the data in the \db.

So far, for you, a \db\ is just a nondescript thing to store data.
But you may have no idea about all the cool things that you could do with a \db.
That you could use a \db\ for your own personal purposes, ranging from keeping track of your finances over managing a bibliography of papers to storing information about your music collection.

Long story short:
We will now explore a small and self-contained example not to teach you how exactly to do things, but to show you what is possible.
And that many things are possible without in-depth knowledge and lots of work.
To make you curious.
To invite you to explore things out of your own interest and to circle back to this book to combine practical experience with background information when you like.
It is important that this is a \emph{learning by doing} example.
When reading the text, please reproduce the example step-by-step on your own computer.

Our example is a \pglspl{db} for a small company that produces shoes and handbags.
You were hired to build an \pgls{IT}~department for the company.
On you first day, your boss enters and tells you \emph{Make a \db\ for storing all our product variants and customer orders.}

We use a concrete technological environment for our work.
In particular, we rely on the \postgresql\ ecosystem.%
%
\usefulTool{postgresql}{%
\postgresql~\cite{TA2024DDAMWPAM,FP2023LP,OH2017PUAR,B2024PELUYDW} is an advanced relational \pgls{dbms}. %
It is free and open source and the basis for all hands-on examples in our course.%
}%
%
