%
\hsection{Accessing the Database from LibreOffice Base}%
%
A completely different way to work with \dbs\ on a professional \dbms\ is to connect to them from \pglspl{GUI}.
Typical examples for such a \pgls{GUI} are the commercial \microsoftAccess\ and the free and open source \libreofficeBase.
Both allow you to create \db\ tables in single files and work on them on your local computer, as they treat \dbs\ as local files for the current user to work on.
As you can infer from the things we have already seen and done, a \inQuotes{real \dbms} offers us the ability to store the \db\ on one compute and to connect to this computer from other computers using multiple \pglspl{client} to work on the centrally managed data.
So neither \microsoftAccess\ nor \libreofficeBase\ can be recommended as \dbms\ for a complex application going beyond simple hobbiest tasks or small-office scenarios.

However, they both offer some pretty cool tools, such as reports and forms.
And instead using them as \dbms, they can be used as \pglspl{GUI} to connect to a \db\ inside another \dbms.
And then we can use these cool tools to work with the \db\ maintained by a professional \dbms.
And that is what we are going to do right now, using the free \libreofficeBase.
Please refer to \cref{sec:installLibreOffice} if you have not yet installed it on your machine.%
%
\FloatBarrier%
\hinput{connect}{connect.tex}%
\hinput{tableAndView}{tableAndView.tex}%
\hinput{erd}{erd.tex}%
\hinput{forms}{forms.tex}%
\hinput{reports}{reports.tex}%
%
\FloatBarrier%
\endhsection%
%
