%
\hsection{Accessing the Database from LibreOffice Base}%
\label{sec:accessingDbFromLibreOfficeBase}%
%
In the previous example, we have discussed how we can connect with \python\ to our \dbms.
In \python, we can develop an application with a nice user interface.
Or we could make a \python\ program that draws data from different sources, like web sites, web services, other \pglspl{db}, sensors in production machines, etc., and insert the data into our \db.
Or maybe we write a \python\ program that analyzes the data from or \db\ using \pgls{AI} and \pgls{ML} to predict future sales.
Or maybe our \python\ code runs inside a web \pgls{server} and offers us a web form to enter data manually.
All of these are rather \inQuotes{big} applications.
They are designed for special purposes by software engineers, implemented and maintained by programmers, and thus cost serious money.
Rarely will they happen in small office situations or small companies.

However, even in less affluent environment, it often makes sense to use \pglspl{db}.
There are many situations where smaller businesses or even individuals have the need to store and process data.
In such cases, they may look for a commercial solution but such solution may not be available for their specific needs or may be too expensive.
Or maybe we are at an early stage of rolling out a \db\ in a bigger organization.
We may plan big investments in the future.
But we first need to convince the organization that there is merit in our project.
We want to have quick and very cheap prototype that allows us to at least enter and access data in our \db.

We already learned how to access \pglspl{db} via \sql\ and how to access them from a programming language.
A third and completely different way to work with \pglspl{db} on a professional \dbms\ is to connect to them from \pglspl{GUI}.
Typical examples for such a \pgls{GUI} are the commercial \microsoftAccess\cite{SSI2023MA2BTA,B2020HOMA2,UC2021AFD} and the free and open source \libreofficeBase\cite{FNFHWSKLSSGLFRSRPLJG2022BG7R1BOL7C,S2022L7PFEUU}.
Both allow you to create \db\ tables in single files and work on them on your local computer.
They realize such \pglspl{db} as local files for the current user to work on.
As you can infer from the things we have already seen and done, a \inQuotes{real \dbms} offers us the ability to store the \db\ on one computer and to connect to this computer from other computers using multiple \pglspl{client} to work on the centrally managed data.
So neither \microsoftAccess\ nor \libreofficeBase\ can be recommended as \dbms\ for a complex application going beyond simple hobbiest tasks or small-office scenarios.

However, they both offer some pretty cool tools, such as reports and forms.
And instead using them as \dbms, they can be used as \pglspl{GUI} to connect to a \db\ inside another \dbms.
And then we can use these cool tools to work with the \db\ maintained by a professional \dbms.
And that is what we are going to do right now, using the free \libreofficeBase.
Please refer to \cref{sec:installLibreOffice} if you have not yet installed it on your machine.%
%
\FloatBarrier%
\hinput{connect}{connect.tex}%
\hinput{tableAndView}{tableAndView.tex}%
\hinput{erd}{erd.tex}%
\hinput{forms}{forms.tex}%
\hinput{reports}{reports.tex}%
%
\usefulTool{libreOfficeBase}{%
\libreofficeBase~\cite{FNFHWSKLSSGLFRSRPLJG2022BG7R1BOL7C,S2022L7PFEUU} offers us a simple \pgls{GUI} that can connect to a \dbms\ and provides capabilities such as executing \sql\ queries as well as designing and executing forms and reports.%
}%
%
\FloatBarrier%
\endhsection%
%
