%
\hsection{Summary}%
%
With this, we have reached the end of our simple introductory example.

What did we learn?
First of all, we got some hands-on experience using one of the world's leading \dbms, \postgresql.
We connected to the \postgresql\ \server\ using the \psql\ \client\ software.
We issued commands in the \sql\ language to the \dbms.

What kind of commands did we issue?
Well, we created a new user (or role), we created a \db, we created tables inside the \db, we issued queries to read data back from tables, we used queries to join data from different tables, we created a view and we built queries on top of that view.
Finally, we deleted everything again.
This means that we have seen several of the most important \sql\ commands that exist.
Surely, we only have played with them.
We are not even close to really understand their full behavior, special cases, performance issues, nor do we have a clear picture of what happens behind the scenes.

But one thing is clear:
If somebody would come and ask us to create a \db\ for some certain application, we could probably do it.
We could stitch together something that does the trick.
Would it be an efficient \db?
Definitely not.
We did not yet learn anything about how to design nice \dbs.
We have no experience whatsoever.

Even more so, we also understand how to access a \db\ from outside, from a programming language like \python.
Based on what we learned, we have a rough feeling of what a \dbms\ can and probably cannot do.
This allows us to, in principle, develop ideas for even more complex applications.
The things that the \dbms\ can do should go into the \db.
The things that it cannot, like displaying forms, processing data from some source like sensors or files, or training \pgls{ML} models on the data would go into the \python\ program code.
The two parts of our application would communicate via a library like \psycopg.

At this stage, I hope that your curiosity is tickled.
Maybe you even have some problems or application ideas for which you might want to design your own \db.
Maybe to catalog your books or music collection, maybe to construct your ancestry tree, maybe to store your bibliographic references.
Nothing will help you more in learning about \dbs\ than doing your own little pet projects.
The second-best thing you can do is to read the rest of this book (or maybe any other book).%
%
\endhsection%
%
