%
\hsection{The Table \inQuotes{demand}}%%
\label{sec:factory:demand}%
%
We now have two tables.
In the first table, we have the products that we can sell.
In the second table, we have a list of customers.
Now we want to store the actual orders, the sales of our company.
We here only consider a very simplistic approach to order management:
With each issued order, a customer can buy a certain amount of exactly one product.
From the real world, you know that usually you can order multiple products in each purchase in an online shop, decide for a payment option, and maybe specify a shipping address different from the address associated with your account.
However, we here keep it plain and simple.%
%
\hinput{create}{create.tex}%
\hinput{insert_and_select}{insert_and_select.tex}%
%
\FloatBarrier%
\endhsection%
%
