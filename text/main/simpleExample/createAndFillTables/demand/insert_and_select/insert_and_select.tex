%
\hsection{Inserting and Selecting some Data}%
%
\gitLoadAndExecSQL{factory:insert_into_table_demand}{}{factory}{insert_into_table_demand.sql}{factory}{boss}{superboss123}%
\listingSQLandOutput{factory:insert_into_table_demand}{%
Storing some order records in the table \textil{demand}.%
}{}%
%
Inserting data into our new table~\sqlil{demand}, however, is a bit annoying.
We need to refer to the customers and the products by their~\sqlil{id}.
In \cref{lst:factory:insert_into_table_demand}, we use an \sqlilIdx{INSERT INTO} command to insert eight orders into the table.
We need to specify the value of \sqlil{customer}, \sqlil{product}, \sqlil{amount}, and the \sqlil{ordered} date for each of them.
The first row is \sqlil{(1, 7, 12, '2024-11-21')}.
It means that customer~1, namely Mr.~Bibbo orders twelve units of product~7, i.e., \inQuotes{Shoe, Size 42.}
He did this on November~21, 2024.
The second row is \sqlil{(2, 3, 2, '2024-12-09')}, meaning that customer~2, i.e., Mr.~Bebbo, ordered 2~units of product~3, namely \inQuotes{Shoe, Size 38.}
This happened on December~9, 2024.
We insert several rows like this.
For instance, row \sqlil{(3, 11, 10, '2025-01-16')} identifies Mrs.~Bebba~(customer~3) as the purchaser of ten large purses~(product~11) on January~16, 2025.

\Cref{lst:factory:insert_into_table_demand} shows the effect of this command.
Querying \sqlil{SELECT * FROM demand;}\sqlIdx{SELECT{\idxdots}FROM} will show us all the rows of this table.
Making sense of this data is, however, not straightforward.
Luckily, \sql\ does not just offer us the ability to link records between tables in order to maintain data integrity.
It also provides us means to connect the information from different tables together.%
%
\endhsection%
%
