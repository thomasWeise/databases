%
\hsection{Inserting some Data}%
\afterpage{\clearpage}%
%
\gitSQLAndOutput{\databasesCodeRepo}{factory}{insert_into_table_product.sql}{factory}{boss}{superboss123}{postgres.sh}{factory:insert_into_table_product}{%
Storing some products in the table~\sqlil{product}.%
}%
%
Now the table \sqlil{product} exists, but it is empty.
Let us fill it with data.
Our factory has two products:~\inQuotes{Shoe} and~\inQuotes{Purse.}
The shoes come in sizes~36 to~43.
Their prices start at 150.99元 for size~36 and increase by 2元 per size.
They all fit into the same box.
The smallest shoes weight 1300g and the weight increases by 25g per size.
Purses come in sizes \emph{small}, \emph{medium}, and \emph{large}, at prices of 100元, 120元, and 150元, respectively.
They weight 500g, 750g, and 1500g, respectively.
The smallest purse fits into a shoebox, but the bigger ones require bigger boxes.

We store this data into the table \sqlil{product} by an \sqlilIdx{INSERT INTO} statement.
Here, we first need to provide the table name~(\sqlil{product}) and the attributes that we want to store in parentheses, i.e., \inQuotes{\sqlil{(...)}}.
We will store values for the fields \sqlil{name}, \sqlil{price}, \sqlil{weight}, \sqlil{width}, \sqlil{height}, and~\sqlil{depth}.
We do not need to store values for~\sqlil{id}, because they will be automatically generated for us.
After saying what we want to store, we specify the \sqlilIdx{VALUES} to store.
Each row is written in parentheses, values and rows are separated by commas.
The complete command for storing all the data is shown in \cref{lst:factory:insert_into_table_product}.

There, we first print all the data currently in the table by typing \sqlil{SELECT * FROM product;}\sqlIdx{SELECT{\idxdots}FROM}\sqlIdx{WHERE}\sqlIdx{*}~\cite{PGDG:PD:SC:S}.
This prints nothing, because the table is empty.
Then we insert the eleven products via one single \sqlilIdx{INSERT INTO} command.
Afterwards, we try \sqlil{SELECT * FROM product;}\sqlIdx{SELECT{\idxdots}FROM}\sqlIdx{WHERE}\sqlIdx{*} again -- and now it prints 11~rows.

Before we continue, let us briefly check what the \sqlilIdx{UNIQUE} constraint that we have defined on column~\sqlil{name} does.
Basically, it says that there cannot be two records in our table with the same value of~\sqlil{name}.
Therefore, if we would try to insert another product with name \sqlil{'Shoe, Size 36'} into the table, this should fail.
Because we already have a row with this value.
We test this in \cref{lst:factory:insert_into_table_product_error}.
Indeed, \cref{exec:factory:insert_into_table_product_error} shows that this fails with an error and the data remains unchanged.%
%
\endhsection%
%
