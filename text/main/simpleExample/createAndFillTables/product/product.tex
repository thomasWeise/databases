\hsection{The Table \inQuotes{product}}%
\label{sec:factory:table:product}%
%
We begin with storing the the information about the products that our company produces and sells.
We want to store all information that may be relevant to customers and the delivery department.
We will give our first new table the name~\sqlil{product}.%
%
\bestPractice{tableName}{%
Table names should be singular nous written in lowercase without any prefix~(i.e., no \inQuotes{tbl\_} in front)~\cite{B2025DS:SBPASG}.%
}%
%
Product data is comprised of different datatypes, ranging from text to numerical values.
Our first table will thus help us to get a glimpse of some of the datatypes supported by \sql.
More importantly, we will learn how to create tables, how to insert data into them, and how to read the data back from the \pgls{db}.%
%
\hinput{create}{create.tex}%
\hinput{insert}{insert.tex}%
\hinput{select}{select.tex}%
%
\FloatBarrier%
\endhsection%
%
