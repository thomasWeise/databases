%
\hsection{Selecting Data}%
%
Now we have stored data in the table \sqlil{product}.
But how can we get it out again?
Well, you already learned a good part of how to do that:~\sqlil{SELECT * FROM product;}\sqlIdx{SELECT{\idxdots}FROM}\sqlIdx{WHERE}\sqlIdx{*}.
This query lists basically all of the data in the table.
You have seen it its output at the bottom of \cref{exec:factory:insert_into_table_product}.

Yet, most often, we do not want to retrieve \emph{all} of the data in a table.
Usually, we only want some part of the data.
Maybe we only want to see the rows~(records) that match certain criteria.
Maybe we only want to see a subset of the columns.
Maybe we even want to compute some statistics.
How can we do that?
A large part of the answer is \inQuotes{With the \sqlil{SELECT}\sqlIdx{SELECT{\idxdots}FROM} statement.}

This is a seduce-to-use example, something to play around with.
So we will play around with the data for a bit in \cref{lst:factory:select_from_table_product}.

First, let's say that we want a list of the names and prices of all types of purses that we sell.
Let us amend the original query \sqlil{SELECT * FROM product;} for this purpose.
The \sqlilIdx{*} here means that all columns should be printed.
Naturally, we would replace it with the columns that we want, namely, \sqlil{name} and \sqlil{price}.
We write \sqlil{SELECT name, price FROM product;}.
This gives us the names and prices of \emph{all} products in our table.
We need to narrow this down to purses.
We can add a \sqlilIdx{WHERE} clause at the end of the query where we can supply a condition.
Only the records that match the condition will be shown.
What condition can we use?
\sql\ offers us some \emph{pattern matching methods}~\cite{PGDG:PD:PM}.
The pattern \sqlil{LIKE '\%Purse\%'}\sqlIdx{LIKE}\sqlIdx{\%} will match any string that contains the text~\inQuotes{Purse}.
The condition \sqlil{name LIKE '\%Purse\%'} therefore requires that the product name contains the text~\inQuotes{Purse}.
Our first real query thus becomes \sqlil{SELECT name, price FROM product WHERE name LIKE '\%Purse\%';}.
As you can see in \cref{exec:factory:select_from_table_product}, it will return three rows of purse-related data.%
%
\begin{sloppypar}%
Assume now that you are a lady who wants to purchase some fashion accessory to accentuate your beauty.
Naturally, you would want to buy the product that gives you the best deal in terms of product weight per monetary unit, i.e, g~per~元.
Therefore, for each product, we would like to divide the weight by the price and give this new value the name \sqlil{g_per_yuan}.
Luckily, \sql\ supports mathematical expressions~\cite{PGDG:PD:MFAO}, so it is possible to write \sqlil{weight / price}.
The query \sqlil{SELECT name, weight / price AS g_per_yuan from product;} would return the product name and the weight-cost ration.
\sqlil{weight / price AS g_per_yuan}\sqlIdx{AS} means that the ratio of weight and price will be computed and given the name \sqlil{g_per_yuan}.%
\end{sloppypar}%
%
\gitLoadAndExecSQL{factory:select_from_table_product}{}{factory}{select_from_table_product.sql}{factory}{boss}{superboss123}%
\listingSQLandOutput{factory:select_from_table_product}{%
Selecting information from the table \textil{product}.}{}%
%
Suppose that our table \sqlil{product} contains hundreds of entries.
It would be very hard to spot good deals in that heap of data.
Luckily, \sql\ also allows to sort data.
We would like to see our list sorted based on \sqlil{g_per_yuan} from large to small values.
This way, the best deals will come first.
We can do that by simply adding \sqlil{ORDER BY g_per_yuan DESC}\sqlIdx{ORDER BY!DESC}.
\sqlilIdx{ORDER BY} sorts the rows of the query result by the fields listed afterwards.
If just write \sqlilIdx{ORDER BY} or, optionally, add \sqlil{ASC}\sqlIdx{ORDER BY!ASC}, then it sorts the data in ascending order.
This means that small values coming first.
We want a descending order, so we also write~\sqlil{DESC}\sqlIdx{ORDER BY!DESC}.

Finally, we may realize that there are still way to many entries returned.
We only care about the best five or so deals, the rest does not matter anyway.
All we have to do to limit the number of rows returned to five is to, well, add~\sqlil{LIMIT 5}\sqlIdx{LIMIT} to our query.
With this, the query is completed.
It is the second one in \cref{lst:factory:select_from_table_product}.

Its result in \cref{exec:factory:select_from_table_product} shows us that the large purse is definitely the best deal here.
For every single~元, we can get 10g of product!
Indeed, the purse weights 1.5kg and costs 150元, so the result is correct.
The second-best deal would be the largest shoe in stock.
At size~43, we can 8.94g of product per~元.

What else can we find out about the data in this table?
What if we wanted to know whether shoes or purses costed more on average?
First, let's figure out how to compute arithmetic means in \sql.
It is rather easy.
We could write \sqlil{SELECT AVG(price) FROM product;}\sqlIdx{AVG} to get the arithmetic mean over all values in the column \sqlil{price} in the table \sqlil{product}~\cite{PGDG:PD:AF}.
This query would return a single row with a single value named \sqlil{avg}.
That value would be \sqlil{148.5381818181818182} if you want to try it by yourself.

Let's give the value a better name.
Let's try \sqlil{SELECT AVG(price) AS mean_price FROM product;}\sqlIdx{AVG}
The result would still be pretty much the same, but now the returned column is named~\sqlil{mean_price}~(and there still only a single row).%
%
\begin{sloppypar}%
As the next step, let's compute the average price for purses.
We can reuse the condition \sqlil{WHERE name LIKE '\%Purse\%'} from before and write \sqlil{SELECT AVG(price) AS mean_price FROM product WHERE name LIKE '\%Purse\%';}
This returns a single column named \sqlil{mean_price} and a single row.
The value in that row is now \sqlil{123.3333333333333333}, which indeed is the arithmetic mean of 100, 120, and~150.
To make clear that this is the purse price, we can simply add an artificial column named~\sqlil{kind} with value~\sqlil{'Purse'}.
\sqlil{SELECT 'Purse' AS kind, AVG(price) AS mean_price FROM product WHERE name LIKE '\%Purse\%';}.
Now we got two columns, \sqlil{kind} and \sqlil{mean_price}, and one row with the values \textil{Purse} and \sqlil{123.3333333333333333}.%
\end{sloppypar}%
%
\begin{sloppypar}%
We can, of course, do the same for shoes.
All we have to do is to replace \sqlil{'\%Purse\%'} with \sqlil{'\%Shoe\%'} and change the \sqlil{kind} column accordingly.
\sqlil{SELECT 'Shoe' AS kind, AVG(price) AS mean_price FROM product WHERE name LIKE '\%Shoe\%';} returns also a single row with values \textil{Shoe} and \textil{157.9900000000000000}.
Indeed, the average price of all of our shoes is 157.99元.%
\end{sloppypar}%
%
So we have two queries that each return two values with the same names.
At this point, we already know that purses are cheaper than shoes in average.
But having two queries is somehow unsatisfying.
We want to package both results together, so that we get the two rows as the result of a single \sql\ command.

Nothing easier than that!
We just have to remove the trailing~\sqlil{;} from the first query and write a \sqlilIdx{UNION ALL} directly in front of the second query~\cite{PGDG:PD:CQUIE}!
The \sqlilIdx{UNION ALL} statement effectively appends the results of the second query to the results of the first query.
The combined statement is shown at the bottom of \cref{lst:factory:select_from_table_product} and its result is given in \cref{exec:factory:select_from_table_product}.

At this point, please notice the beauty of queries:
We can continue to add data to our table.
We can change values or delete values.
But the queries will still work all the same and always give us the up-to-date results.%
%
\FloatBarrier%
\endhsection%
%
