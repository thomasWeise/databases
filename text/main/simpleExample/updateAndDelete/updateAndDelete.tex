%
\hsection{Updating and Deleting Records}%
%
We have seen how new data can be inserted into a \db\ row-by-row in \cref{sec:factoryCreatingTableAndInsertingData}.
Often, we also want to modify or delete data.
This can be done similarly easily.
%
\hsection{Updating Records}%
Assume that our factory found another producer for shoe boxes.
The new boxes are 5mm less high and therefore cheaper.
In \cref{lst:factory:update_table_product} we want to change the package size for all products in shoe boxes accordingly.
We do this with an \sqlil{UPDATE <table> SET <fields>}\sqlIdx{UPDATE{\idxdots}SET} statement~\cite{PGDG:PD:U}.%
%
\sqlSyntax{syntax/update.sql}%
%
\gitLoadAndExecSQL{factory:update_table_product}{}{factory}{update_table_product.sql}{factory}{boss}{superboss123}%
\listingSQLandOutput{factory:update_table_product}{%
Modifying some records in the table \sqlil{product}.%
}{}%
%
\gitLoadAndExecSQL{factory:update_table_product_error}{}{factory}{update_table_product_error.sql}{factory}{boss}{superboss123}%
\listingSQLandOutput{factory:update_table_product_error}{%
The attempt to update the table~\sqlil{product} in a way that would violate the referential integrity:~%
We cannot change the primary key \sqlil{id} value of a record that is referenced by a foreign key from table~\sqlil{demand}.%
}{}%
%
The table is clearly \sqlil{product}.
There is only a single field that we want to change, namely \sqlil{height}.

The shoe box that we used so far were 350mm $\times$ 250mm $\times$ 130mm in size.
For each product that ships in such a box, we want to change the height as \sqlil{height = height - 5}, i.e., we set the \inQuotes{new} height to be the \inQuotes{old} height minus 5mm.
We could write \sqlil{height = 245} just as well, but in the example I wanted to show that we can use arbitrary expressions when updating column.

However, we do not want to change all the rows in the table.
Only those that used the shoe boxes of the aforementioned dimensions.
So we use an \sqlilIdx{WHERE} condition the selects only the rows for which\sqlIdx{AND} \sqlil{width = 350}, \sqlil{height = 250}, and \sqlil{depth = 130} hold.

Finally, we want to see the result of our update query.
The \sqlil{UPDATE} statement, as implemented by \postgresql, allows us to supply, basically, a \sqlil{SELECT} query that chooses data only from the \emph{changed} rows\footnote{This is a \postgresql\ addition to~\sql~\cite{SE:DA:2020WDSSOAOTNPSDIOR} and does not seem to be part of the \sql\ standard, but it is also supported by \sqlite~\cite{HWACIS:R}.}.
This query is written at the end of the statement as \sqlilIdx{RETURNING}.
We return all the data from the changed rows.

As you can see \cref{exec:factory:update_table_product}, nine rows are affected.
Their \sqlil{height} indeed changed to~\sqlil{245}.

When we designed the table~\sqlil{demand}, we create two foreign key columns:
\sqlil{customer} and \sqlil{product}.
The latter one references the rows in table~\sqlil{product} by their corresponding~\sqlil{id} values.
We did learn that it is impossible to insert a row into the table~\sqlil{demand} whose value in column~\sqlil{product} does not occur somewhere in column~\sqlil{id} of table~\sqlil{product}.
In other words, we cannot have a \sqlil{demand}~record that stored the sale of a product that does not exist.
If we create a \sqlil{demand}~record, then there must be corresponding and existing row in~\sqlil{product} associated with it.
Otherwise, the record creation failes, as we demonstrated in \cref{exec:factory:insert_into_table_demand_error_1}.

What, however, happens if we change the value of the \sqlil{id} of an existing product?
In the very first row in table~\sqlil{demand}, customer Bibbo orders 12 units of product~\inQuotes{Shoe, Size 42}.
\inQuotes{Shoe, Size 42} has \sqlil{id = 7}.
Therefore, that \sqlil{demand} record stores value~\sqlil{7} in its \sqlil{product}~attribute.

What will happen if we change the value of~\sqlil{id} of the record with~\sqlil{id = 7} in table~\sqlil{product} to, say,~\sqlil{id = 20}?
If we do this, then all records in the table~\sqlil{demand} which hold \sqlil{product = 7} would become invalid.
Because there no longer would be any row in table~\sqlil{product} with~\sqlil{id = 7}.
But this is another table, right?
If we change table~\sqlil{product}, will this be affected by the constraints that we defined upon \emph{a completely different} table~(namely table~\sqlil{demand})?
In \cref{lst:factory:update_table_product_error} we try to do exactly that.
And we cannot!
\Cref{exec:factory:update_table_product_error} clearly shows that, no, we cannot violate the referential integrity that way.
We cannot change the \sqlil{id} value of a record in table~\sqlil{product} \emph{if} that record is referenced elsewhere via a foreign key~(\sqlilIdx{REFERENCES}) constraint.%
\endhsection%
%
\hsection{Deleting Records}%
%
%
After running our company for some time, we realized that our stock of \inQuotes{Shoe, Size 36} is never declining.
Indeed, nobody ever purchased these smallest-size shoes.
Brokenheartedly, we decide to discontinue this product.
This means that we somehow need to remove it from the table \sqlil{product}, as it will no longer be sold.

In \cref{lst:factory:delete_from_table_product}, we do this with the command \sqlil{DELETE FROM product WHERE id = 1;}\sqlIdx{DELETE FROM}\sqlIdx{WHERE}~\cite{PGDG:PD:D}.
This command is pretty self-explanatory:%
%
\sqlSyntax{syntax/delete.sql}%
%
\gitLoadAndExecSQL{factory:delete_from_table_product}{}{factory}{delete_from_table_product.sql}{factory}{boss}{superboss123}%
\listingSQLandOutput{factory:delete_from_table_product}{%
Deleting a row from the table \sqlil{product}.%
}{}%
%
\gitLoadAndExecSQL{factory:delete_from_table_customer_error}{}{factory}{delete_from_table_customer_error.sql}{factory}{boss}{superboss123}%
\listingSQLandOutput{factory:delete_from_table_customer_error}{%
We try to delete a record in table~\sqlil{customer} that is used in records in table~\sqlil{demand}. %
This fails, because otherwise, the referential integrity of our data would be violated.%
}{}%
%
In \cref{lst:factory:delete_from_table_product}, we use it to delete all the rows from table \sqlil{product} where the field~\sqlil{id} has value~\sqlil{1}.
Only one such row exists, namely the one with \inQuotes{Shoe, Size 36}.
It is deleted.
Under \postgresql, we can directly access the deleted rows via the \sqlilIdx{RETURNING} statement\footnote{This is a \postgresql\ addition to~\sql~\cite{SE:DA:2020WDSSOAOTNPSDIOR} and does not seem to be part of the \sql\ standard, but it is also supported by \mariadb~\cite{M:MSD:D} and \sqlite~\cite{HWACIS:R}.}.
To see whether this really worked, we print \sqlil{SELECT COUNT(*) as number_of_products from product;} before and after the \sqlilIdx{DELETE FROM} query.
Indeed, \cref{exec:factory:delete_from_table_product} shows that we originally had 11~products in our palette.
After deleting the smallest-sized shoes, only ten products are left.

We know that changing records via \sqlil{UPDATE} is not permitted if it would lead to a violation of referential integrity.
How about deletion of records?
Let's say we want to delete the record of customer Bebbo, which has \sqlil{id = 2}.
This customer issued several orders in our system, so there are several rows in table~\sqlil{demand} that reference this record via their foreign key~\sqlil{customer}.
In \cref{lst:factory:delete_from_table_customer_error}, we attempt to delete Mr.~Bebbo from our \db.
And it fails, as it should, because then we would have orphaned records in our table~\sqlil{demand}.
We could, however, first delete all of these records from table~\sqlil{demand}~(while leaving unrelated records intact, of course).
Then we could delete Mr.~Bebbo's \sqlil{customer} record afterwards.%
%
\endhsection%
%
\hsection{A Note on Compatibility}%
In this section, we learned a lot of useful things.
We learned how we can update and delete data from tables.
This basically completes our understanding of the most important operations in \sql.

However, we also touched another important subject:
While most \pglspl{rdb} support the \sql\ language, many of them still differ in the features that they implement and may even add own statements and extensions.
\postgresql, for example, provides the \sqlilIdx{RETURNING} clause in its \sqlilIdx{INSERT}, \sqlilIdx{UPDATE}, and \sqlilIdx{DELETE} statements~\cite{SE:DA:2020WDSSOAOTNPSDIOR}.
This is a very very useful clause, because we often want to know, e.g., which data has been changed, or which automatically generated~\sqlil{id}\nobreakdashes-value has been assigned to a row upon insertion.
This clause is only supported by some other \pglspl{dbms} and even then maybe only partially~\cite{HWACIS:R,M:MSD:D,M:MSD:IR}.
Many \pglspl{dbms} offer the same basic functionality, but may differ in features and command syntax.
So even if \sql-based \pglspl{rdb} seem to be compatible at first glance, most often they are not.
Choosing a \pgls{dbms} for a project therefore is often a decision that cannot easily be changed later and, thus, has to be deliberated carefully.%
\endhsection%
\FloatBarrier%
\endhsection%
%
