%
\hsection{Updating and Deleting Records}%
%
\gitLoadAndExecSQL{factory:update_table_product}{}{factory}{update_table_product.sql}{factory}{boss}{superboss123}%
\listingSQLandOutput{factory:update_table_product}{%
Modifying some records in the table \sqlil{product}.%
}{}%
%
\hsection{Updating Records}%
%
We have seen how new data can be inserted into a \db\ row-by-row in \cref{sec:factoryCreatingTableAndInsertingData}.
Often, we also want to modify or delete data.
This can be done similarly easily.

Assume that our factory found another producer for shoe boxes.
The new boxes are 5mm less higher and therefore cheaper.
In \cref{lst:factory:update_table_product} we want to change the package size for all products in shoe boxes accordingly.
We do this with an \sqlil{UPDATE <table> SET <fields>}\sqlIdx{UPDATE{\idxdots}SET} statement~\cite{PGDG:PD:U}.
The table is clearly \sqlil{product}.
There is only a single field that we want to change, namely \sqlil{height}.

The shoe box that we used so far were 350mm $\times$ 250mm $\times$ 130mm in size.
For each product that ships in such a box, we want to change the height as \sqlil{height = height - 5}, i.e., we set the \inQuotes{new} height to be the \inQuotes{old} height minus 5mm.
We could write \sqlil{height = 245} just as well, but in the example I wanted to show that we can use arbitrary expressions when updating column.

However, we do not want to change all the rows in the table.
Only those that used the shoe boxes of the aforementioned dimensions.
So we use an \sqlilIdx{WHERE} condition the selects only the rows for which\sqlIdx{AND} \sqlil{width = 350}, \sqlil{height = 250}, and \sqlil{depth = 130} hold.

Finally, we want to see the result of our update query.
The \sqlil{UPDATE} statement allows us to supply, basically, a \sqlil{SELECT} query that chooses data only from the \emph{changed} rows.
This query is written at the end of the statement as \sqlilIdx{RETURNING}.
We return all the data from the changed rows.

As you can see \cref{exec:factory:update_table_product}, nine rows are affected.
Their \sqlil{height} indeed changed to~\sqlil{245}.
\endhsection%
%
\hsection{Deleting Records}%
%
\gitLoadAndExecSQL{factory:delete_from_table_product}{}{factory}{delete_from_table_product.sql}{factory}{boss}{superboss123}%
\listingSQLandOutput{factory:delete_from_table_product}{%
Deleting a row from the table \sqlil{product}.%
}{}%
%
After running our company for some time, we realized that our stock of \inQuotes{Shoe, Size 36} is never declining.
Indeed, nobody ever purchased these smallest-size shoes.
Brokenheartedly, we decide to discontinue this product.
This means that we somehow need to remove it from the table \sqlil{product}, as it will no longer be sold.

In \cref{lst:factory:delete_from_table_product}, we do this with the command \sqlil{DELETE FROM product WHERE id = 1;}\sqlIdx{DELETE FROM}\sqlIdx{WHERE}~\cite{PGDG:PD:D}.
This command is pretty self-explanatory.
It deletes all the rows from table \sqlil{product} where the field~\sqlil{id} has value~\sqlil{1}.
Only one such row exists, namely the one with \inQuotes{Shoe, Size 36}.
It is deleted.

To see whether this really worked, we print \sqlil{SELECT COUNT(*) as number_of_products from product;} before and after the \sqlilIdx{DELETE FROM} query.
Indeed, \cref{exec:factory:delete_from_table_product} shows that we originally had 11~products in our palette.
After deleting the smallest-sized shoes, only ten products are left.%
%
\endhsection%
\endhsection%
%
