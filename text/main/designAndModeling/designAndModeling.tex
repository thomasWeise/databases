\hsection{Database Design and Modeling}%
%
In the introductory example in the previous book part, we had some idea about how a \db\ should look like and then just implemented it.
Of course, this is not how that works.
In the real world, \dbs\ are much more complicated.
There are not just three tables.
The interactions between objects, processes, and people are much more complex.
While we now do have some rough idea about what kind of technological tools we have available to work with \dbs, we have very little understanding of the practical process of \db\ design.

Let us pick up again on the example of running a factory and a shop.
Let us try to make this scenario more realistic this time.
We are the young and energetic and motivated new hire of the factory.
So far, they manage all the customers, products, the storehouse, everything, by using paper-based lists and registers.
As the very first person with a computer science background here, it is our job to create a software that can manage the complete state of the company and ensure that all information is consistent and clear.%
%
\hinput{lifecycle}{lifecycle.tex}%
%
\endhsection%
%
