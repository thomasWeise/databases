\hsection{Database Design and Modeling}%
%
In the introductory example in the previous book part, we had some idea about how a \db\ should look like and then just implemented it.
Of course, this is not how that works.
In the real world, \dbs\ are much more complicated.
There are not just three tables.
The interactions between objects, processes, and people are much more complex.
While we now do have some rough idea about what kind of technological tools we have available to work with \dbs, we have very little understanding of the practical process of \db\ design.

Let us now do a second example on \db\ design.
This time, we will look a bit more closely at the process and the steps involved in creating a reasonably elaborate \db\ application.
We want to design a \db\ for managing students, teachers, and courses in a university.%
%
%
\hinput{lifecycle}{lifecycle.tex}%
\hinput{requirements}{requirements.tex}%
%
\endhsection%
%
