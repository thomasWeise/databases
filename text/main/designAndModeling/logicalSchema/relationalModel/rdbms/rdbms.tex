\hsection{Relational Database Management Systems}%
\label{sec:rdbms}%
%
In the 1980s, many vendors of \pgls{dbms} did not completely implement the relational data model as developed by \citeauthor{C1985IYDRR}.
Instead, they added mechanics that circumvent the relational characteristics, either because of laziness, to allow backwards compatibility to older systems, or in order to improve performance.
In a response, \citeauthor{C1985IYDRR} defined thirteen rules that govern a relational~\pgls{dbms}~\cite{C1985IYDRR,C1986AESFDBMSTACTBR,S2024D:LDMRMRA,SP2002STYSI2D,SPMP1998SI2TDDASVEI12T}.
Since the first rule is called \emph{Rule~0}, the thirteen rules are referred to as \emph{the twelve rules}.%
%
\begin{enumerate}%
\setcounter{enumi}{-1}%
%
\item Foundation Rule:~A relational \pgls{dbms} must be able to manage \pglspl{db} entirely through its relational capabilities.%
%
\item Information Rule:~All information in a relational database is represented explicitly at the logical level and in exactly one way -- by values in tables.
This includes even table names, column names, and column types, which, too, must be stored in a table.
Such special tables that store the structure of a \db\ are usually part of the built-in system catalog.
This system catalog holds the metadata of the system and is~(or is part of) a relational \db\ itself.

Remember back when we first began working with \postgresql\ in our very first, very simple example?
In \cref{sec:simpleExample:createUser}, we started by creating a new user for the \pglspl{dbms} called \sqlil{boss}.
We checked the list of existing users by writing \sqlil{SELECT usename FROM pg_catalog.pg_user}.
\sqlil{pg_catalog.pg_user} is the name of a table, which belongs to the system catalog.
When we connected to our \db\ using \libreofficeBase, we saw lots of strange tables in \cref{fig:factoryLibreOfficeBaseConnect12openingScreen}.
These belong to the system catalog.%
%
\item Guaranteed Access Rule:~Each and every datum~(atomic value) in a relational \db\ is guaranteed to be accessible via a combination of table name, primary key value, and column name.%
%
\item Systematic Treatment of Missing Values Rule:~\sqlilIdx{NULL} values~(distinct from the empty character string or a string of blank characters, and distinct from zero or any other number) are supported in fully relational \pgls{dbms} for representing missing information and inapplicable information in a systematic way~(independent of data type).

This rule has been a point of arguments over meany years~\cite{C20245YOQ}:
Real data does include unspecified elements.
There may be street addresses without house number, there may be people without phone number.
So there is a need to represent such situations.
However, having unspecified or missing values also violates the definition of tuples in relations.
We could imagine that the domain of each attributes contains the additional value~\sqlilIdx{NULL}, too, though.%
%
\item Dynamic Online Catalog based on the Relational Model:~The \db\ description is represented at the logical level just like ordinary data, so that authorized users can apply the same relational language to its interrogation as they apply to the regular data.%
%
\item Comprehensive Data Sublanguage Rule:~A relational system must support at least one language whose statements are expressible per some well-defined syntax as character strings; and which supportings all of the following items:%
\begin{enumerate}%
\item data definition~(e.g., creating tables),%
\item view definition~(creating views),%
\item data manipulation~(e.g., adding and deleting of data),%
\item integrity constraints~(e.g., limits on data range, foreign keys, \dots),%
\item authorization~(e.g., user managenment), and%
\item transaction boundaries~(begin, commit, and rollback).%
\end{enumerate}
In the case of our book, this language is \sql.
Of course, one could conceive and support also other languages.%
%
\item View Updating Rule:~All views that are theoretically updatable are also updatable by the system.%
%
\item Insert, Update, and Delete Rule:~The capability of handling a base relation or a derived relation as a single operand applies not only to the retrieval of data but also to the insertion, update, and deletion of data.
In other words, the operations do not just apply to single records~(rows) but can concern multiple rows at once, because their inputs can be relations~(whole tables, results of \sqlil{SELECT}, \dots).%
%
\item Physical Data Independence Rule:~Application programs and terminal activities remain logically unimpaired whenever any changes are made in either storage representations or access methods.
In other words, the way the \pgls{dbms} actually stores the data has no impact on how an application accesses data via the text-based language.%
%
\item Logical Data Independence Rule:~Application programs and terminal activities remain logically unimpaired when information-preserving changes of any kind that theoretically permit unimpairment are made to the base tables.
Let's say we split a table into two tables and distribute the rows into either part, leaving columns and primary keys intact.
We can then design a view that merges the two tables~(using \sqlilIdx{UNION}).
An application sitting on top of that will not see any change.%
%
\item Integrity Independence Rule:~Integrity constraints specific to a particular relational database must be definable in the relational data sublanguage and storable in the catalog~(not in the application programs).
We did this several times, for example with the \sqlilIdx{PRIMARY KEY} constraint, the \sqlilIdx{REFERENCES} constraint, and the \sqlilIdx{CHECK}\sqlIdx{CONSTRAINT!CHECK} constraint as far back as in \dref{sec:factoryCreatingTableAndInsertingData}.
All of them were defined in \sql.
%
\item Distribution Independence Rule:~A relational \pgls{dbms} has distribution independence.
This means that a \pgls{dbms} \emph{may support} storing data distributed over several different computers~(nodes) or a cluster.
\emph{If} the \pgls{dbms} supports such distribution, then this should not affect the \sql\ programs sent to it.
In this case, the \pgls{dbms} must take care of dividing the queries to the corresponding nodes and re-assemble the results.
If a \pgls{dbms} does not support distribution of data over a network, then it automatically fulfills this rule.%
%
\item Non-Subversion Rule:~If a relational system has an additional low level~(single-record-at-a-time) language, that low level cannot be used to subvert or bypass the integrity rules and constraints expressed in the higher level relational language~(multiple-records-at-a-time).
In other words, a \pgls{dbms} must support at least one relational language~(Rule~5), but it may support any other access languages or programming interfaces as well, some of which may work on single records, some may not be relational and so on.
However, none of these access methods must be allowed to violate the integrity of the relational data.%
%
\end{enumerate}%
%
These rules are implemented to a large degree by modern \pglspl{dbms}.
They also tell us relatively exactly what to expect, with what kind of features we can work.%
%
\endhsection%
%
