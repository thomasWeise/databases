\hsection{The Relational Data Model}%
\label{sec:relationalModel}%
%
In the \pglspl{ERD} that we painted \emph{before} \cref{sec:compactCrowsFootNotation}, there were three visual components:
entity types~(rectangles), attributes~(ellipses), and relationship types~(diamonds).
When we moved to the more compact visualization style in \cref{sec:compactCrowsFootNotation}, the relationship diamonds disappeared.
Instead, they were represented just by straight lines.
This has two reasons:
First, the relationship diamonds waste space.
Second, in the relational data model, relationships do not exist as independent objects.
In this model, we only have entity types~(embodied by tables) and attributes~(the columns of the tables).
Relationships are realized as foreign keys, i.e., as special attributes, and as constraints.%
%
\hinput{definitions}{definitions.tex}%
\hinput{keys}{keys.tex}%
\hinput{rdbms}{rdbms.tex}%

We now have a basic understanding about how the relational data model works.
We know the basic definitions.
Even better, we can tie the definitions that we learned for conceptual modelling into the definitions for the relational model.
We find that they are quite similar.
We also learned the basic requirements for a relational \dbms.
Combining these information with what we already learned in the initial example of this book, we have a pretty clear understanding of what relational \pglspl{dbms} like \postgresql\ offer to us.
And we may have some good ideas about how we can transform our conceptual models to relational ones tied to an \sql\ \db.
Which, coincidentally, is what we will do next.%
\endhsection%
%
