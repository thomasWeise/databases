\hsection{Definitions}%
In the context of \pglspl{rdb}, the same definitions for attributes and domains~(\cref{def:attribute,def:attributeDomain}) that we already discussed back in \dref{sec:entitisAttrsErd} are used.
The following additional definitions are commonly considered~\cite{C1970ARMODFLSDB}:%
%
\begin{definition}[Relation Schema]%
\sloppy%
\label{def:relationSchema}%
A \emph{relation schema}~\pgls{relSchema} is the ordered sequences of $n$~attributes~$(a_1, a_2, \dots, a_n)$, i.e., is a sequence of attribute names and domains.%
\end{definition}%
\fussy%
%
\begin{definition}[Relation]%
\label{def:rdb:relation}%
A \emph{relation}~$R$ is a set of $n$\nobreakdashes-tuples $R\subseteq\attrDomainb{a_1}\times\attrDomainb{a_2}\times\dots\times\attrDomainb{a_n}$ to which a relation schema~\relSchemab{R} that specifies the attributes~$(a_1, a_2, \dots, a_n)$ is associated~\cite{SS2005EIDDDFDB:SDWSD2}.%
\end{definition}%
%
The definition of relation schemas in the relational model is therefore somewhat equivalent to the definition of entity types in the entity model~(see \cref{def:entityType}).
When translating our conceptual model to a logical relational model, an entity type will become a relation schema.
The difference to the conceptual is that, in the logical schema, we will use relations to implement both entities and relationships.

Also, at first glance, one may think that \inQuotes{Relations = Tables} in a \db.
In other words, one may think that relations are implemented as tables.
But this is only partially true:
Relations can also be the result from a \sqlil{SELECT}\sqlIdx{SELECT{\idxdots}FROM} statement in \sql.
Relations can also be the parameter of an \sqlilIdx{INSERT INTO} statement.
Thus, relations are a quite versatile concept to represent our data.

Notice that a relation is a \emph{set} of tuples.
Since a set cannot contain the same element twice, this means that duplicate tuples~(rows, records) are not permitted in relations by definition~\cite{C20245YOQ}.
As a deviation from the pure formalism, the \sql\ language does permit duplicate tuples in tables and query results~\cite{C20245YOQ}.
Sets are also not ordered, so there is no default order of the tuples in relations either.

All attributes~(columns) must have names, i.e., there are no anonymous attributes~\cite{S2024D:LDMRMRA}.
In the original works on \pglspl{rdb}~\cite{C1970ARMODFLSDB}, the order of the attributes~(columns) in a relation mattered and it was permitted that two column have the same name.
This idea was later abandoned.
Today, the order of columns are unimportant and the columns of a table must have unique names~\cite{S2024D:LDMRMRA}.
The values of attributes are atomic, i.e., there are no multivalued attributes and no composite attributes~\cite{S2024D:LDMRMRA,SS2005EIDDDFDB:SDLDUTRDM}.

The degree of a relation is defined as follows~(please to not mix this up with the degree of a \emph{relationship} discussed in \cref{def:degreeOfRelationship}):%
%
\begin{definition}[Degree of a Relation]%
The \emph{degree} of a relation is the number~$n$ of its attributes.%
\end{definition}%
%
Relations are at the core of \pglspl{rdb}.%
%
\cquotation{C1970ARMODFLSDB}{%
The totality of data in a data bank may be viewed as a collection of time-varying relations. %
These relations are of assorted degrees. %
As time progresses, each $n$\nobreakdashes-ary relation may be subject to insertion of additional $n$\nobreakdashes-tuples, deletion of existing ones, and alteration of components of any of its existing $n$\nobreakdashes-tuples.}%
%
\endhsection%
