%
We will now work our way through the ten different types of relationships between two entity types that can occur in an \pgls{ERD} created during conceptual modeling.
Back in \cref{sec:conceptual:relationshipCardinalities}, we already painstakingly worked our way through these types and tried to find examples in existing sources.
This was nothing.
We will now set out to find how each of these relationships can be implemented in \sql.

We will do this in plain \sql\ and not in the \pgmodeler, because we start from the visual representation of the relationships and want to transform them to \sql.
Using \pgmodeler, we would practically do the same, just in a convenient \pgls{GUI}.
\pgmodeler\ is also more suitable for managing larger models, whereas we will slash and hammer our way through several small models with two entity types each.

Please also consider this as an exercise in \sql.
This is not so much about whether all of these relationship types do occur in practice.
It is also not about memorizing the different approaches how they can be implemented.
It is mainly about getting some feeling and understanding how the utilities that \sql\ offers us, mainly \sqlil{NOT NULL}, \sqlil{REFERENCES}, \sqlil{UNIQUE}, and \sqlil{PRIMARY KEY}~constraints~\cite{PGDG:PD:C} together with \sqlil{INNER JOIN} queries~\cite{PGDG:PD:JT} can be used to enforce referential integrity between tables.
And also, for some relationship types {\dots} it is even fun to figure out how they can be implemented.

Of course, keeping with our practical \inQuotes{This is what it looks like when we execute it on the \postgresql\ \pgls{server}.}~attitude, we spin up a \db\ to really see some of the concepts in action in~\cref{lst:conceptualToRelational:init}.%
%
\gitSQLAndOutput{\databasesCodeRepo}{conceptualToRelational}{init.sql}{}{}{}{postgres.sh}{conceptualToRelational:init}{%
We spin up a \db\ for running our example \sql\ codes when mapping conceptual relationship between entity types to tables in~\sql\ on the \postgresql\ \gls{server}.%
}%
%
\FloatBarrier%
