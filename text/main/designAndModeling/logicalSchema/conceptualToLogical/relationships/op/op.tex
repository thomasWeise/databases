%
\hsection{\crowsFoot{O}{OM}{P}{OM}}%
\label{sec:rm:op}%
%
\gitSQLAndOutput{\databasesCodeRepo}{conceptualToRelational}{OP_tables.sql}{relationships}{}{}{postgres.sh}{OP_tables}{%
The realization of a \crowsFoot{O}{OM}{P}{OM} conceptual relationship.%
}%
\gitSQLAndOutput{\databasesCodeRepo}{conceptualToRelational}{OP_insert_and_select.sql}{relationships}{}{}{postgres.sh}{OP_insert_and_select}{%
Inserting into and selecting data from the realization of an \crowsFoot{O}{OM}{P}{OM} conceptual relationship given in \cref{lst:OP_tables}.%
}%
\gitSQLAndOutput{\databasesCodeRepo}{conceptualToRelational}{OP_insert_error.sql}{relationships}{}{}{postgres.sh}{OP_insert_error}{%
Trying to relate two entities twice, which is of course not permitted in \emph{any} relationship pattern.%
}%
%
We have the two entity types~O and~P.
Each entity of type~O may be connected to zero, one, or multiple entities of type~P.
Each entity of type~P may be connected to zero, one, or multiple entities of type~O.

We need a table~\sqlil{o} for the entities of type~O.
The primary key of this table be~\sqlil{oid} and there also will be the example attribute~\sqlil{x}.
We also need a table~\sqlil{p} for the entities of type~P, which gets the primary key~\sqlil{pid} and the example attribute~\sqlil{y}.
Now, since each row in~\sqlil{o} can be related to multiple rows in table~\sqlil{p} and vice versa, no two-table solution is possible.
This time, there is no way around it:
We need three tables.
But this three-table approach will look pretty much like the one in \cref{sec:rm:ab} for the \crowsFoot{A}{O1}{B}{O1}, but with different~\sqlil{UNIQUE} constraints.

In \cref{lst:OP_tables}, we add the table \sqlil{relate_o_and_p} which has two columns,~\sqlil{fkoid} and~\sqlil{fkpid}, which are foreign keys pointing to the primary keys~\sqlil{oid} and~\sqlil{pid} of tables~\sqlil{o} and~\sqlil{p}, respectively.
This is ensured with corresponding \sqlil{REFERENCES} constraints.
Both columns are marked as~\sqlil{NOT NULL}.
Neither of them is~\sqlil{UNIQUE}, because each row of table~\sqlil{o} can be related to multiple rows of table~\sqlil{p} and vice versa.
Still, the a pair~\sqlil{(fkoid, fkpid)} can appear only once in the table, because two specific rows in tables~\sqlil{o} and~\sqlil{p} can, of course, be related only once to each other.
We could enforce this with a constraint~\sqlil{UNIQUE (fkoid, fkpid)}.
However, the right solution here is to go directly for~\sqlil{PRIMARY KEY (fkoid, fkpid)}.
Our table needs a primary key, and here, the only possible primary key are the pairs of~\sqlil{(fkoid, fkpid)}.
And primary keys must be unique by default, so this constraint also covers the uniqueness.

In \cref{lst:OP_insert_and_select}, we now insert data into the two tables.
Since both relationship ends are optional, we can first enter some data into the tables~\sqlil{o} and~\sqlil{p}.
Then we can add the relationships between the rows of these tables by inserting rows into table~\sqlil{relate_o_and_p}.
If we want to recombine data from the two tables, we can do this with two \sqlil{INNER JOIN}~expressions.

During the above example, we inserted the row~\sqlil{(1, 1)} into table~\sqlil{relate_o_and_p}.
This row establishes that the row with primary key~\sqlil{oid = 1} of table~\sqlil{o} is related to the row with primary key~\sqlil{pid = 1} in table~\sqlil{p}.
In \cref{lst:OP_insert_error}, we try inserting the row again into~\sqlil{relate_o_and_p}.
Of course, no two rows can be related twice.
(It would make, for example, no sense to assign the same address twice to the same person.)
Thanks to the \sqlil{PRIMARY KEY} constraint that we attached to table~\sqlil{relate_o_and_p}, this insertion fails.%
%
\FloatBarrier%
\endhsection%
%
