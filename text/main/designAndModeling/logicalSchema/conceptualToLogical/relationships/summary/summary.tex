%
\hsection{Summary}%
%
\gitLoadAndExecSQL{conceptualToRelational:cleanup}{}{conceptualToRelational}{cleanup.sql}{}{}{}%
\listingSQLandOutput{conceptualToRelational:cleanup}{cleanup.sql}{%
We now tear down the example \db\ that we used to play around with when mapping conceptual relationships between entity types to tables in~\sql.%
}{}%
%
This was quite a journey.
We have worked our way through every single binary conceptual relationship type that can be expressed with crow's foot notation.
We listed all of these relationship types back in \dref{sec:conceptual:relationshipCardinalities}.

We found that each of these types can be enforced in a relational \dbms.
With \emph{enforced}, we refer to the cardinality and modality of the relationship ends.
If we want to implement, for example, a relationship following the pattern \crowsFoot{K}{M1}{L}{OM} in a \dbms, then we can do that.
We can create a table~\sqlil{k} for the entities of type~K and a table~\sqlil{l} for the entities of type~L.
Then we can use constraints that enforce that each row in table~\sqlil{l} must be related to exactly one row in table~\sqlil{k}.
We can enforce that, at no time, there can ever be a row in table~\sqlil{l} that is not related to a row in table~\sqlil{k} or related to more than one such row.
We can also enforce that each row in table~\sqlil{k} can be related to an arbitrary amount of rows in table~\sqlil{l}.
Of course, we also enforce that if a row in table~\sqlil{l} is related to one row in table~\sqlil{k}, then that row in table~\sqlil{k} must be related to that row in table~\sqlil{l} and vice versa.
No relationship end can be \inQuotes{open}.
If these conditions are met, then our \db\ has the property of \emph{referential integrity}.

What we also learned is that specifying the necessary constraints is not always easy.
Especially if both relationship ends are mandatory, things can get complicated.
We still managed to find solutions for all cases and also learned some additional \sql\ commands, such as~\glsreset{CTE}\pglspl{CTE}.
Sometimes, however, we used \postgresql-specific extensions to \sql.
One particularly useful one is the \sqlil{RETURNING} statement~\cite{PGDG:PD:RDFMR}, which is also supported by \mariadb~\cite{M:MSD:IR} and \sqlite~\cite{HWACIS:R}.
Another one were sequences and the \sqlil{NEXTVAL} function~\cite{PGDG:PD:CS,PGDG:PD:SMF}.

Due to the complexity, we may sometimes choose to change the relationship types when moving from the conceptual model to the logical model.
We only considered binary relationship between two entity types, and doing something like \crowsFoot{M}{M1}{N}{MM} while also having \crowsFoot{N}{MM}{X}{MM} might prove too annoying to actually implement.

Creating examples for every single possible relationship type for two tables was fun, though.
As final step, we can now get rid of our example \db\ in~\cref{lst:conceptualToRelational:cleanup}.%
%
\endhsection%
