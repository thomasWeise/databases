%
\hsection{\crowsFoot{S}{MM}{T}{MM}}%
\label{sec:rm:st}%
%
\gitSQLAndOutput{\databasesCodeRepo}{conceptualToRelational}{ST_tables.sql}{relationships}{}{}{postgres.sh}{ST_tables}{%
The realization of a \crowsFoot{S}{MM}{T}{MM} conceptual relationship.%
}%
\gitSQLAndOutput{\databasesCodeRepo}{conceptualToRelational}{ST_insert_and_select.sql}{relationships}{}{}{postgres.sh}{ST_insert_and_select}{%
Inserting into and selecting data from the realization of an \crowsFoot{S}{MM}{T}{MM} conceptual relationship given in \cref{lst:ST_tables}.%
}%
\gitSQLAndOutput{\databasesCodeRepo}{conceptualToRelational}{ST_insert_error_1.sql}{relationships}{}{}{postgres.sh}{ST_insert_error_1}{%
Trying to insert new related rows into tables~\sqlil{s} and~\sqlil{t} without updating table~\sqlil{relate_s_and_t} does not work.%
}%
\gitSQLAndOutput{\databasesCodeRepo}{conceptualToRelational}{ST_insert_error_2.sql}{relationships}{}{}{postgres.sh}{ST_insert_error_2}{%
Trying to insert a new row into tables~\sqlil{s} and relate it to an existing row in table~\sqlil{t} without updating table~\sqlil{relate_s_and_t} does not work either.%
}%
%
We have the two entity types~S and~T.
Each entity of type~S must be connected to at least one entity of type~T, but can be connected to many.
Each entity of type~T must be connected to at least one entity of type~P, but can be connected to many.

To implement this relationship pattern, we will combine what we learned when implementing the \crowsFoot{M}{M1}{N}{MM}~pattern in \cref{sec:rm:mn} with the method for implementing the \crowsFoot{Q}{OM}{R}{MM}~pattern in \cref{sec:rm:qr}.

We first again create the basic tables that we are definitely going to need in \cref{lst:ST_tables}.
We need a table~\sqlil{s} for the entities of type~S.
The primary key of this table be~\sqlil{sid} and there also will be the example attribute~\sqlil{x}.
We also need a table~\sqlil{t} for the entities of type~T, which gets the primary key~\sqlil{tid} and the example attribute~\sqlil{y}.%
%
\begin{sloppypar}%
We will also definitely need a third table to manage the relationships, which we will call~\sqlil{relate_s_and_t}.
This table has two columns,~\sqlil{fksid} and~\sqlil{fktid}, which are foreign keys pointing to the primary keys~\sqlil{sid} and~\sqlil{tid} of tables~\sqlil{s} and~\sqlil{t}, respectively.
This is ensured with corresponding \sqlil{REFERENCES} constraints.
Both columns also are marked as~\sqlil{NOT NULL}, because neither value can be omitted in a row.
Like in \cref{sec:rm:op}, each pair~\sqlil{(fksid, fktid)} can appear only once in the table, because two specific rows in tables~\sqlil{s} and~\sqlil{t} can, of course, be related only once to each other.
This is implemented via the constraint~\sqlil{PRIMARY KEY (fksid, fktid)}.%
\end{sloppypar}%
%
Like in the \crowsFoot{M}{M1}{N}{MM}~pattern, both relationship ends are mandatory.
We cannot create a row in table~\sqlil{s} without relating it to an existing row in table~\sqlil{t}.
We also cannot create a row in table~\sqlil{t} without relating it to an existing row in table~\sqlil{s}.
Back in \cref{sec:rm:mn}, we solved this chicken-and-egg problem by creating a sequence from which we could then generate values for the primary key of the table~\sqlil{m}.
We would use this sequence to first create a primary key, use the primary key as foreign key in the table~\sqlil{n}, and use the primary key of that row together with the one for table~\sqlil{m} to finally insert a new row in table~\sqlil{m}.
We will follow a similar approach here, so we first invoke~\sqlil{CREATE SEQUENCE sqsid AS INT;}.
The primary key for table~\sqlil{s} is then defined as~\sqlil{sid INT DEFAULT NEXTVAL('sqsid') PRIMARY KEY}.

Both tables~\sqlil{s} and~\sqlil{t} will have a column referencing a primary key from the respective other table.
For table~\sqlil{s}, this is column~\sqlil{fktid}, and for table~\sqlil{t}, this is column~\sqlil{fksid}.
These will be used to enforce that each row in table~\sqlil{s} is definitely related to one row in table~\sqlil{t} and vice versa.
Of course, they can also be related to multiple rows, which is why we need to manage the relationships in table~\sqlil{relate_s_and_t}.%
%
\begin{sloppypar}%
We also must make sure that for each row in table~\sqlil{s} with data~\sqlil{(sid, fktid)}, there exists a corresponding row~\sqlil{(fksid, fktid)} in table~\sqlil{relate_s_and_t}.
We do this by adding the constraint~\sqlil{s_sid_fktid_fk} which is~\sqlil{FOREIGN KEY (sid, fktid)} that~\sqlil{REFERENCES relate_s_and_t (fksid, fktid)} to table~\sqlil{s}.
This is the exactly same approach we used for the \crowsFoot{Q}{OM}{R}{MM}~pattern in \cref{sec:rm:qr}.%
\end{sloppypar}%
%
The difference is that we must do this also for table~\sqlil{t}, because both relationship ends are mandatory.
In other words, we add the constraint~\sqlil{t_fksid_tid_fk} to table~\sqlil{t} that ensures that every tuple~\sqlil{(fksid, tid)} in that table must also appear as tuple~\sqlil{(fksid, fktid)} in table~\sqlil{relate_s_and_t}.

Defining the constraints that enforce referential integrity for pattern is one thing, finding a way to insert data into the tables under these tight constraints is another issue.
In \cref{lst:ST_insert_and_select}, we do that.

Initially, the tables are empty.
This means that we need to create three rows at once:
We need to create a row in table~\sqlil{s} and we must immediately relate it to a newly created row in table~\sqlil{t} \emph{and} this relationship must also appear as row in table~\sqlil{relate_s_and_t}.
Thanks to \pglspl{CTE}, this is possible.

First, we create a new primary key value for table~\sqlil{s} by creating the \pgls{CTE}~\sqlil{s_id} corresponding to \sqlil{SELECT NEXTVAL('sqsid') AS new_sid}.
We then use this new primary key value when inserting a row into table~\sqlil{t} via~\sqlil{INSERT INTO t (y, fksid) SELECT 'AB', new_sid FROM s_id}.
Of course, we do this as another \pgls{CTE} named~\sqlil{new_t} also doing~\sqlil{RETURNING tid, fksid}.
This means that this \pgls{CTE} will provide is the primary key of the new row in table~\sqlil{t} and the primary key~\sqlil{fksid} that we already allocated for the row in table~\sqlil{s} that we will create next.
And now we create this row.
We invoke \sqlil{INSERT INTO s (sid, x, fktid) SELECT fksid, '123', tid FROM new_t}.
Notice how this uses the pre-created primary key as, well, primary key.
It also uses the primary key of the new row in table~\sqlil{t} as foreign key.
This is going to be our third \pgls{CTE}, called~\sqlil{new_s}, which also returns both keys via~\sqlil{RETURNING sid, fktid}.
Finally, we can insert a row into~\sqlil{relate_s_and_t} by doing \sqlil{INSERT INTO relate_s_and_t (fksid, fktid) SELECT sid, fktid FROM new_s}.
Thanks to \pglspl{CTE}, we could insert one row in three tables each, in a single \sql\ command, which performs the referential integrity checks at its end.%
%
\begin{sloppypar}%
Now there are existing records in the tables~\sqlil{s} and~\sqlil{t}.
It is comparatively easy to create a new row for table~\sqlil{s} that is related to an existing row in table~\sqlil{t}.
In this case, all we need to do is to insert the row into table~\sqlil{s} and, at the same time, insert a row into table~\sqlil{relate_s_and_t}.
We just need a single~\pgls{CTE}, which we will call~\sqlil{new_s}, and which performs \sqlil{INSERT INTO s (x, fktid) VALUES ('456', 1) RETURNING sid, fktid}.
As result, we get the primary key~\sqlil{sid} of the new row in table~\sqlil{s} as well as the foreign key to table~\sqlil{t}, \sqlil{fktid}, which is the same as the one we provided when creating the row in table~\sqlil{s}, namely~\sqlil{1}.
We can then \sqlil{INSERT INTO relate_s_and_t (fksid, fktid) SELECT sid, fktid FROM new_s} and are done.%
\end{sloppypar}%
%
Since the relationship pattern is symmetric, we can do exaclty the same for table~\sqlil{t}.
We can insert a new row into table~\sqlil{t} and relate it to an existing row in table~\sqlil{s}.
For this, we would procede exactly the same way and also use one~\pgls{CTE}.

Finally, we can also simply relate two existing rows in tables~\sqlil{s} and~\sqlil{t} by just creating one new row in table~\sqlil{relate_s_and_t}.
This can be like this:~\sqlil{INSERT INTO relate_s_and_t VALUES (1, 3);}.

Merging the data requires again two \sqlil{INNER JOIN} expressions, exactly as before.
The constraints prevent us from creating rows in~\sqlil{s}~(or~\sqlil{t}) that are not related to rows in~\sqlil{t}~(or~\sqlil{s}).
We also cannot relate rows without creating the corresponding entry in~\sqlil{relate_s_and_t}, as shown in \cref{lst:ST_insert_error_1,lst:ST_insert_error_2}.%
%
\FloatBarrier%
\endhsection%
%
