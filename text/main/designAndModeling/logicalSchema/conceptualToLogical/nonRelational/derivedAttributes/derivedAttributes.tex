\hsection{Derived Attributes}%
%
Derived attributes, introduced in \cref{def:derivedAttribute}, are attributes that are computed based on the values of other attributes.
Whether or not they should be stored in a table depends on how hard it is to compute them.
For example, imagine that we have a factory and package things in boxes, pretty much like back in \dref{sec:factory:table:product}.
Maybe we have a table in our \db\ just for storing the types of boxes by their width, height, and depth.
The volume of a box could be derived attribute, computed as the product of the three dimensions.
It would not make sense wasting space by storing this value, because we can easily compute it in an \sql\ query.

Then again, we may sell multiple products as packages.
For example, we may offer a vacuum cleaner together with a spare battery, two different nozzles, two brushes, an edge tool, and other attachments, and a pack of air filter papers.
These package elements need to all fit together into a box and the order in which they are packed into the is important.
This is a bit like a puzzle game.
The best order could be computed based on the dimensions of the box and the dimensions of the package components.
Doing so, however, means solving an \npHard\ problem, which requires lots of time.
Then, we would definitely store the corresponding data as attributes even though they could be re-computed.

A derived attribute that is stored is just a normal table column.
If the value can be computed from values of other columns, as is the case in the box volume example above, the column can be annotated as \sqlilIdx{GENERATED}~\cite{PGDG:PD:GC}.
At the time of this writing, \postgresql~17 is the current major version.
This major version only supports \sqlilIdx{STORED} generated columns~\cite{PGDG:PD:GC}, but in the future, \sqlilIdx{VIRTUAL} generated columns whose values are not stored in the \db, but which are computed when read, will be introduced.

In summary, there are three choices of how we can implement derived attributes:%
%
\begin{itemize}%
%
\item realize them as \sqlilIdx{GENERATED} table columns and store their values~\cite{PGDG:PD:GC},%
\item realize them as virtual \sqlilIdx{GENERATED} table columns without storing their values, which is currently not yet supported by \postgresql~\cite{PGDG:PD:GC}, but will be implementedd in the future and may be supported by other \dbms\ already, or%
\item realize them by computing their values in \sqlilIdx{SELECT} queries as needed.%
%
\end{itemize}%
%
Now, in the conceptual model illustrated as \cref{fig:erdPerson1B} in the previous section, there was one derived attribute, namely~\emph{Age}.
This attribute is supposed to represent the age of a person.
Of course, if we know the \pgls{dateOfBirth} of a the person, we can compute the age right away.
So how would we realize this attribute:
As a stored \sqlilIdx{GENERATED} column or compute it on the fly during \sqlilIdx{SELECT} queries?

Storing this value makes no sense:
The age of a person is not a constant.
It will change as time goes by.
Also, even if \postgresql\ does support virtual (not stored) \sqlilIdx{GENERATED} columns, it would \emph{still} be a mistake to use those to represent the age of people.%
%
\cquotation{PGDG:PD:GC}{%
Several restrictions apply to the definition of generated columns and tables involving generated columns:\\%
\textbf{-}~The generation expression can only use immutable functions and cannot use subqueries or reference anything other than the current row in any way.\\%
\textbf{-}~\dots%
}%
%
To compute the age of a person, we need the \pgls{dateOfBirth} and we will somehow subtract it from the \sqlil{CURRENT_DATE}\sqlIdx{CURRENT\_DATE}.
\sqlil{CURRENT_DATE}\sqlIdx{CURRENT\_DATE} certainly is not an immutable or constant function.
It changes daily.
%
\gitLoadAndExecSQL{person_database_2:view_person_age}{}{teachingManagement/logical/person_database_2}{view_person_age.sql}{person_database}{}{}%
\listingSQLandOutput{person_database_2:view_person_age}{view_person_age.sql}{%
Create a view that represents the derived attribute \emph{age} and execute it.%
}{}%
%
\gitExecSQLraw{}{}{teachingManagement/logical/person_database_2}{cleanup.sql}{}{}{}%

So regardless of whether we have virtual \sqlilIdx{GENERATED} columns or not, the best way to compute the age of a person is by defining a query, maybe solidified as a \sqlilIdx{VIEW}.
Still having the \sqlil{person_database} from the previous section lying around, we will do it in there.

In \cref{lst:person_database_2:view_person_age}, we create a view\sqlIdx{CREATE!VIEW}\sqlIdx{VIEW} with the name \sqlil{person_age}.
\postgresql\ offers the function \sqlilIdx{AGE}~\cite{PGDG:PD:DTFAO} which computes the difference between \sqlil{CURRENT_DATE}\sqlIdx{CURRENT\_DATE} and the \sqlilIdx{TIMESTAMP} or \sqlilIdx{DATE} value provided as parameter.
Our new view simply selects all the columns from table~\sqlil{person} and adds a new column called~\sqlil{age} computed as~\sqlil{AGE(date_of_birth)}.
In \cref{exec:person_database_2:view_person_age}, we then first print the value of the \sqlil{CURRENT_DATE}\sqlIdx{CURRENT\_DATE} of the time when this book is compiled for reference.
Then we combine the result of the new view \sqlil{person_age} with the table~\sqlil{name} to print all the person records together with their name, \pglspl{dateOfBirth}, and ages.
Having done that, we can delete the \db\ \sqlil{person_database} again.%
%
\endhsection%
%
