\hsection{Other Non-Relational Objects}%
%
When we begin designing an application, we develop the conceptual model.
For this purpose, we usually use \glsreset{ERD}\pglspl{ERD}.
\Pglspl{ERD} give us lots of freedom in how we can represent the real world.
We can use strong entities, weak entities, and relationships, and all three of these object types can have attributes.
Attributes can be single-values or multi-valued as well as atomic or composite.

We then use the relational data modelling approach for the logical model of our applications.
At its core, the only type of objects offered by relational model for storing data are relations.
Relations have attributes which are single-valued and atomic.
Tables are the practical implementation of relations inside a \dbms\ and table columns represent attributes.

One of the beautiful things of the relational data model is that it feels very natural.
Many components of the conceptual modeling level can directly be translated to the logical level.
For example, Entities become tables.

Conceptual relationships are objects in the conceptual model that do not exist in the relational model as singular objects.
Instead, they become either tables or columns and constraints, as we have discussed in the previous section.
Multi-valued attributes, which are useful at the conceptual level, also do not exist in the relational model.
They become tables in their own right.
Composite attributes are broken down to their elements which then become columns.
All of this we have already discussed.

When scrolling over our section on conceptual modelling, we find that three types of objects are still \inQuotes{left over:}%
%
\begin{enumerate}%
%
\item We discussed strong entities, but not weak entities.%
%
\item We discussed relationships, but not relationship attributes.%
%
\item While we exhaustively discussed all possible binary relationship patterns, we did not discuss relationships of a higher degree, i.e., situations where three or more entity types participate in a relationship.%
%
\end{enumerate}%
%
To complete our treatment on the translation of conceptual models to logical models, let us also take a look how these are translated to the relational data model.%
%
\FloatBarrier%
\hinput{weakEntities}{weakEntities.tex}%
\hinput{relationshipAttributes}{relationshipAttributes.tex}%
%
\endhsection%
%
