\hsection{First Normal Form}%
\label{sec:normalForm:1}%
%
The \glsreset{1NF}\pgls{1NF} dates back to \citeauthor{C1970ARMODFLSDB}'s seminal paper~\cite{C1970ARMODFLSDB} where he presented the relational data model back in~\citeyear{C1970ARMODFLSDB}.
Under the \pgls{1NF}, all rows in a table must have the same number of fields and all fields must be atomic.
This excludes multivalued attributes as well as composite attributes.
As we already discussed before the relational data model does not support such attributes anyway.

In \cref{sec:mappingEntitiesToTables}, we discussed how entity types in the conceptual schema are translate to tables in the logical schema based on the requirements of the relational data model.
We stated that multivalued attributes become separate tables and that composite attributes need to be recursively broken down into their atomic components, which then become separate columns.
If we use the relational data model, then we would naturally produce logical models in \pgls{1NF}.

However, this is only true if we \emph{recognize} multivalued attributes and composite attributes as such.
If a table does have attributes that are semantically composite but we implement them as single flat attributes, then it violates the \pgls{1NF}.
If an attribute of an entity is multivalued, but instead of placing it into an additional table, we try to represent it using multiple columns in the table for entity, we violate the \pgls{1NF}.
Let us explore what happens if we violate the \pgls{1NF}.%
%
\FloatBarrier%
\hinput{composite}{composite}%
\hinput{multivalued}{multivalued}%
\FloatBarrier%
\endhsection%
%
