\hsection{Example:~Student and Parent Information}%
%
Since, at first glance, the definition is again somewhat unclear, we explore it with an example.
Imagine that we are working on the design of the teaching management platform.
A colleague mentions that it would be a good idea to also have the contact information of one parent stored for each student.
Sometimes there can be situations where such information could be useful.
Maybe a student has an accident on campus.
Maybe a student suddenly stops coming to lectures and is nowhere to be found.
Then it would be good to be able to call a parent.

Assume that the following information about students should be stored in our \db:%
\begin{itemize}%
\item their university-assigned Student~ID~\sqlil{student_id},%
\item their name~\sqlil{student_name},%
\item the name~\sqlil{parent_name} of one of their parents, and%
\item the mobile phone number~\sqlil{parent_mobile} of that parent.%
\end{itemize}%
%
The following \pglspl{funcDep} exist at first glance:%
%
\begin{itemize}%
\item \funcDepb{\sqlil{student_id}}{\sqlil{student_name}}, %
\item \funcDepb{\sqlil{student_id}}{\sqlil{parent_name}}, %
\item \funcDepb{\sqlil{student_id}}{\sqlil{parent_mobile}}, and %
\item \funcDepb{\sqlil{parent_mobile}}{\sqlil{parent_name}}.
\end{itemize}%
%
\hinput{violation}{violation}%
\hinput{fixed}{fixed}%
%
\endhsection%
%
