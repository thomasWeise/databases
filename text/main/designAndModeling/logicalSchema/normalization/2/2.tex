\hsection{Second Normal Form}%
\label{sec:normalForm:2}%
%
We already learned about the \pgls{1NF}.
It will not suprise you that this is not the only normal form.

The \glsFull{2NF} deals with the relationship between key attributes and non-key attributes~\cite{C1971FNOTDBRM,C1971NDBSABT,K1983ASGTFNFIRDT,D2003AITDS,EN2015FODS}.
It applies to composite keys only, i.e., keys that consist of multiple columns of a table.
Back in \cref{def:key,def:key2}, we learned that a key can be used to uniquely identify entities.
If a key~$X$ uniquely identifies an entity, then this means that all the other attributes of the entity basically provide additional information about that key.
In a table in the relational model, a key is a unique identifier for each row.
In other words, there can be at most one row for each value of~$X$.
All the other columns provide additional information about the real-world object represented by that key.

The \pgls{2NF} is violated when a non-key attribute is a fact about a \emph{proper subset} of \emph{any} key~\cite{K1983ASGTFNFIRDT}.
A proper subset can only exist of a set with more than one element.
Therefore, the \pgls{2NF} is only relevant if our tables have a key that is composite, i.e., consists of more than one column.
A table is in \pgls{2NF} if it is in the \pgls{1NF} \emph{and} all columns that are not part of a key provide information about the complete key(s).
Another perspective on the \pgls{2NF} is offered by \glsreset{funcDep}\pglspl{funcDep}.%
%
\begin{definition}[Functional Dependency]%
\label{def:functionalDependency}%
A \glsreset{funcDep}\emph{\pgls{funcDep}} is a relationship between two \emph{groups} of attributes~$X$ and $Y$, such that for each instance of~$X$, the value of~$X$ determines the value of~$Y$~\cite{S2024D:RNDAFDNF}. %
This can be written as~$\funcDepb{X}{Y}$.%
\end{definition}%
%
In other words, if $\funcDepb{X}{Y}$, then there are no two records with the same value of~$X$ but different values of~$Y$~\cite{K1983ASGTFNFIRDT}.
A given value of the key~$X$ must always occur with the same associated value of~$Y$.
Also, if $X$ is a key, then all other columns are by definition dependent on~$X$, simply because there cannot be two rows in a table with the same value of~$X$.

The relational schema of relation~$R$ be~\relSchemab{R} and a key be~$X\subseteq\relSchemab{R}$.
Of course, all attributes~$a$ in~\relSchemab{R} depend on the key attributes~$X$, i.e., it always holds that~$\funcDepb{X}{a}$.
Let us approach a definition of the \pgls{2NF} using \pglspl{funcDep}.
If the \pgls{2NF} is observed, then for all attributes~$a\in\relSchemab{R}$ that are not part of the key~$X$~(meaning $a \not\in X$), there does \emph{not} exist a proper subset~$X'\subset X$ such that~$a$ functionally depends on~$X'$, i.e.~$\funcDepb{X'}{a}$.
(\emph{Proper} means that $X'\neq X$.)
In the \pgls{2NF}, all attributes depend on the \emph{complete} key~$X$.

However, this definition is not fully correct.
As we discussed before and will discuss again later, there can be multiple keys, i.e., multiple sets of attributes, that uniquely identify rows in~$R$.
Yes, we do choose one as primary key, but there may be more keys.
This is especially true if we choose a surrogate primary key because the other candidate keys are too large.
The non-key attributes must depend on all of these keys in their entirety.
Under the \pgls{2NF}, it is not permitted that they depend any subset of any key.
The correct definition for the \pgls{2NF} there is given as follows~\cite{SS2005EIDDDFDB:SDWSD2}:%
%
\begin{definition}[\glsreset{2NF}\Pgls{2NF}]%
\label{def:2nf}%
A relation~$R$ with the primary key~$P\subseteq\relSchemab{R}$ is in the \glsreset{2NF}\pgls{2NF} if and only if it is in the \pgls{1NF} and for all sets of attributes~$X\subseteq\relSchemab{R}$ and all attributes~$a\in\relSchemab{R}$ with~$a\not\in X$, $a\not\in P$ and~$\funcDepb{X}{a}$, it holds that $X$ is either a key or a super key, but not a proper subset of any key of~$R$.%
\end{definition}%
%
This definition may not be entirely clear at first glance.
Let us do an example to understand it better.%
%
\hinput{rooms}{rooms.tex}%
\hinput{summary}{summary.tex}%
%
\endhsection%
%
