\hsection{Second Normal Form}%
\label{sec:normalForm:2}%
%
The \glsreset{2NF}\pgls{2NF}~\cite{C1971FNOTDBRM,C1971NDBSABT,K1983ASGTFNFIRDT,D2003AITDS,EN2015FODS} applies to composite primary keys only, i.e., primary keys that consist of multiple columns of a table.

Back in \cref{sec:primaryKey}, we learned that a primary key, let's call it~$X$, is used to identify entities.
If a primary key~$X$ uniquely identifies an object, then all the other attributes of an entity provide information about that object.
In a table in the relational model, the primary key is a unique identifier for each row.
In other words, there can be at most one row for each value of~$X$.
All the other columns provide additional information about the real-world object represented by that primary key.
The \pgls{2NF} is violated when a non-key field is a fact about a \emph{subset} of a key~\cite{K1983ASGTFNFIRDT}.
It is only relevant when the key is composite, i.e., consists of several fields.
In other words, a table is in \pgls{2NF} if it is in the \pgls{1NF} \emph{and} all columns that are not part of the primary key provide information about the complete primary key.

Another perspective on the \pgls{2NF} is offered by \glsreset{funcDep}\pglspl{funcDep}.%
%
\begin{definition}[Functional Dependency]%
\label{def:functionalDependency}%
A \glsreset{funcDep}\emph{\pgls{funcDep}} is a relationship between two groups of attributes~$X$ and $Y$, such that for each instance of~$X$, the value of~$X$ determines the value of~$Y$~\cite{S2024D:RNDAFDNF}. %
This can be written as~$\funcDepb{X}{Y}$.%
\end{definition}%
%
In other words, if $\funcDepb{X}{Y}$, then it is invalid to have two records with the same value of~$X$ but different value of~$Y$~\cite{K1983ASGTFNFIRDT}.
A given value of~$X$ must always occur with the same value of~$Y$.
If $X$ is a key, then all fields are by definition dependent on~$X$, because there cannot be two rows in a table with the same value of~$X$.

The relational schema of relation~$R$ be~\relSchemab{R} and the primary key be~$X\subseteq\relSchemab{R}$.
Of course, all attributes~$a$ in\relSchemab{R} depend on the primary key attributes~$X$, i.e., it always holds that~$\funcDepb{X}{a}$.
If the \pgls{2NF} is observed, then for all attributes~$a\in\relSchemab{R}$ that are not part of the primary key~($a \not\in X$) there does not exist a proper subset~$X'\subset X$ with $X'\neq X$ such that~$a$ functionally depends on~$X'$, i.e.~$\funcDepb{X'}{a}$.
In other words, in the \pgls{2NF}, all attributes depend on the \emph{complete} primary key~$X$.%
%
\hinput{violation}{violation.tex}%
%
\endhsection%
%
