\hsection{Conceptual Schema Design}%
\label{sec:conceptualSchemaDesign}%
%
Let us now begin with the conceptual modelling of our application.
From the requirements analysis, we know the entities of the student management platform and how they interact.
However, so far, we discussed them only very informally.
It is now time to put them together into consistent models.
At the conceptual design step, these models will be independent from any concrete technology.
This step is called \emph{entity relationship modeling}~\cite{G2011EW2ITDS:CMUTERM,SS2005EIDDDFDB:CDDRAAML,SS2005EIDDDFDB:CDDICAMP,V1999C5DMS:CDUTERM}.

It should be noted that creating such models makes a lot of sense for larger \db\ applications like our example here.
However, there are also many possible smaller situations where we may want to use a \db, say, to manage our literature reference, to manage a collection of books or musical records.
For such smaller projects, one may directly skip this step and move on to the logical schema~\cite{S2024D:CDMERDE}.
Either way, we are now working on a beautiful and big project, so we definitely want to take this step.
%
\hinput{entitiesAndAttributes}{entitiesAndAttributes.tex}%
\hinput{keys}{keys.tex}%
\hinput{relationships}{relationships.tex}%
%
\endhsection%
%
