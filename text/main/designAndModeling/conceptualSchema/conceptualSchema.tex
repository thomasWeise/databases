\hsection{Conceptual Schema Design}%
\label{sec:conceptualSchemaDesign}%
%
Let us now begin with the conceptual modelling of our application.
From the requirements analysis, we know the entities of the student management platform and how they interact.
However, so far, we discussed them only very informally.
It is now time to put them together into consistent models.
At the conceptual design step, these models will be independent from any concrete technology.
This step is called \emph{entity relationship modeling}~\cite{G2011EW2ITDS:CMUTERM,SS2005EIDDDFDB:CDDRAAML,SS2005EIDDDFDB:CDDICAMP,V1999C5DMS:CDUTERM}.

It should be noted that creating such models makes a lot of sense for larger \db\ applications like our example here.
However, there are also many possible smaller situations where we may want to use a \db, say, to manage our literature reference, to manage a collection of books or musical records.
For such smaller projects, one may directly skip this step and move on to the logical schema~\cite{S2024D:CDMERDE}.
Either way, we are now working on a beautiful and big project, so we definitely want to take this step.
%
\hsection{Entities and Attributes}%
%
The first major component to model are the datastructures to be stored inside the \db.
For this purpose, the following modeling primitives have emerged.%
%
\hsection{Modeling Primitives}%
%
The most basic elements for conceptual modeling are entities, attributes, and entity types.%
%
\begin{definition}[Entity]%
An \emph{entity} is an object or thing with an independent existence in the world. %
It can be distinguished from all other objects.%
\end{definition}%
%
Examples of entities are maybe the student Mr.~Bibbo, the module~\citetitle{programmingWithPython}~\cite{programmingWithPython}, the professor Mrs.~Bebba~教授, room~\#36~305, or the curriculum \emph{Computer Science and Technology}.
Entities can be spotted easily in the requirement specification or when viewing the meeting or interview notes:
They correspond to \emph{proper nouns}~\cite{S2024D:CDMERDE}, i.e., nouns that actually name one specific thing and that are usually capitalized~\cite{EOWM2025MWAMTD:CAPNWTDLWOGC}.%
%
\begin{definition}[Attribute]%
An \emph{attribute} is a feature or characteristic of an entity.%
\end{definition}%
%
Entities in our model are represented by the values of their attributes.
A student could be defined by their name, ID, student~ID, mobile phone number, email address, home address, date of birth, etc.
A module could be described by its title, syllabus, and abstract.
Additionally to such features, \emph{adjectives} in the requirements text often can be interpreted as attribute values, e.g., red, young, successful, heavy, fast.%
%
\begin{definition}[Domain]%
The \emph{domain} of an attribute is the set of possible values that it can take on.%
\end{definition}%
%
The name of a student is a text string.
The mobile phone numbers and IDs may be a text strings, too, but of a fixed length and limited to certain character ranges at certain positions.
The date of birth, on the other hand, is a date.
The score in an exam may be an integer number between~0 and~100.%
%
\begin{definition}[Entity Type]%
\label{def:entityType}%
The set of all entities that have the same attributes is called an \emph{entity type}.%
\end{definition}%
%
So while the entity Mr.~Bebbo is a single student, the set of all possible students would form an entity type.
While the module~\citetitle{programmingWithPython}~\cite{programmingWithPython} is a single entity, the set of all possible modules would form an entity type.
Mrs.~Bebba~教授 is a single entity, but the set of all possible professors represents an entity type.
Entity types are \emph{common nouns} that stand for groups or types of things~\cite{S2024D:CDMERDE} and that are usually written with lowercase letters~\cite{EOWM2025MWAMTD:CAPNWTDLWOGC}.

Also notice the plural \emph{attributes} in \cref{def:entityType}.
Entity types in a \db\ are usually characterized by multiple attributes.
It makes little sense, for example, to consider \emph{year} as an entity type, because it does not have multiple attributes.%
%
\begin{definition}[Entity Set]%
An \emph{entity set} is a subset of an entity type. %
It is a set of some entities of a type that exist at one point in time.%
\end{definition}%
%
For example, Mr.~Bebbo is a single student entity, the set of all possible student entities forms an entity type, but the students Mr.~Bebbo, Mr.~Bibbo, and Mr.~Bobbo together form an entity set.
The modules~\citetitle{programmingWithPython}~\cite{programmingWithPython} and \citetitle{databases}~\cite{databases} form an entity set, because they are a subset of an entity type for modules.
Notice that the mathematical notion of \emph{set} is indeed correct here:
All entities have a unique identity and, hence, can be differentiated from all possible other objects.
There are no two identical Mr.~Bibbos.
Therefore, students can be grouped in a set and entity types are sets, too.

Viewing the conceptual design of \dbs\ through this lense, we notice a few things.
First, we always model only a tiny window to the real world.
When we talk about students, modules, and curricula as entity types, this only concerns our particular application.
Of course, in our real big wide world, students exist in other universities and in other countries.
These students may have completely different attributes from ours.
They do not matter in the model of our small part of the world.

Second, if you attended or do attend a course on programming, then you will feel that this way of modeling things is a bit related to \pgls{OOP}.
Entity types could be thought of as \pythoniles{class}, entities could be their instances and attributes could be their, well, attributes.
While \db\ theorists may dislike this way of thinking, I believe that it is not wrong.
It is a viable analogy.
However, later, when we model the relationships between entities, it may no longer be helpful.%
\endhsection%
%
\hsection{Entitiy Relationship Diagrams}%
We now know that entity types with their attributes basically correspond to datastructures in programming.
They form one element of the conceptual model.
But how do we actually write them down?
How do we specify them?

For this, a graphical notation has been introduced:
\glsreset{ERD}\Pglspl{ERD} are the most commonly used tool to model the entity types and their relationships in a \db~\cite{C2002ERMHEFTALL,C1975TRMTAUVOD,C1976TERMTAUVOD,KW2012ASHOTEDAIM,WF1995DHQDM,B1990CMERMO}.
There exists a wide variety of graphical notations that can be used for \pglspl{ERD}.
The original notations by \citeauthor{B1969DSD}~\cite{B1969DSD} and \citeauthor{C1975TRMTAUVOD}~\cite{C1975TRMTAUVOD,C1976TERMTAUVOD} are still in use, the more comprehensive and standardized \pgls{IDEF1X} syntax~\cite{FIPSPUB184,ISOIECIEEE2012ITMLP2SASFII}, and the \glsreset{UML}\pgls{UML}~\cite{OMG2017OUMLOU,RMHOSMUUIIIIOPPTRS1997UNG,BRJ1999TUMLRM}.%
%
\endhsection%
\endhsection%
%
%
\endhsection%
%
