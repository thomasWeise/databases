\hsection{Database Model Selection}%
\label{sec:dbModelSelection}%
%
Between the design of the conceptual schema and the design of the logical schema, we have to choose a \emph{data model} for our \db~\cite{SS2005EIDDDFDB:I}.%
%
\begin{definition}[Data Model]%
The \emph{data model} specifies the notation and language for defining types (the \db\ schema) and for accessing and updating the \db.%
\end{definition}%
%
There are several important data models.

We could treat our data as single or weakly-interrelated tables, which can be stored in formats such as \pgls{CSV}, \pgls{microsoftExcel}, or \pgls{libreofficeCalc}.
Such formats can deal with thousands of similarly-structured records and perform efficient calculations on them.
They are not suitable for data where records of different types are related and constraints are placed upon such relationships.
They are also not suitable for scenarios where multiple users concurrently work on the same documents.

Another choice would be hierarchically structured single documents, in formats such as \pgls{XML}, \pgls{JSON}, and \pgls{YAML}.
These formats are able to deal with thousands of records that are structured similarly or differently.
They are not suitable for data where records of different types are related and constraints are placed upon such relationships.
They are also not suitable for scenarios where multiple users concurrently work on the same documents.
Still, some open source software such as~\href{https://basex.org}{BaseX}~\cite{GHS2010B,G2006PXMMDTTL} exists that brings more of a \db\ flavor into this area.

The early developments in the \db\ field saw hierarchical \pglspl{db}, such as IMS~\cite{KLBGNLWBS2012ITIYCGTIIMS,BBP2007TBOI,KC2024DS:ITD}.
These offer the advantages of hierarchically structured documents combined with the ability to support concurrent access.
The hierarchical structure of data may create redundancy:
If a professor teaches two classes, then the data about the professor is stored twice.
This approach to \pglspl{db} is basically outdated.
Still, some open source hierarchical \pglspl{dbms} still exist, including MUMPS~\cite{O2008TMPL,O2025TMPL}.
The hierarchical key-value store YottaDB is also an implementation of MUMPS~\cite{B2018THALOMMATFOY}.

Another legacy approach is the network data model, which emerged from IDS~\cite{B2009TOOTIDSITFDAD,B1965SFRAP,H2016HCBITDAFOODW} and CODASYL~\cite{TF1976CDBMS}.
There still exist CODASYL and COBOL-based \pglspl{dbms}~\cite{O2022OCDDARM}.
However, as is the case for hierarchical \pglspl{db}, it may generally be better to implement a \db\ in the relation model.
On one hand, this model fits well to many application domains.
If you can express your data in an hierarchical or network model, you probably can do this with a relational model, too.
On the other hand, there just exists infinitely more and well-maintained open source and commercial relational \pglspl{dbms}.

Another data model are key-value stores, which are basically schema-less and appropriate if your data has a key-value structure.

The main focus of this book, however, is on the relational data model by \citeauthor{C1970ARMODFLSDB}~\cite{C1970ARMODFLSDB} and \sql.
Here, each entity set is represented as relation and stored in a table inside the \db.
As we already practically explored, the tables can also be related and constraints can ensure that the relationships between them remain in a consistent state.
This is what we decide to use for our teaching management platform as well.%
%
\endhsection%
%
