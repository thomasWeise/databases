\hsection{Summary}%
At this stage, we have learned several things about entities and attributes.
An entity models some real-world object, person, thing, location, event, or concept.
An entity does not just exist.
It is the merger of its attribute values.

Attributes can have different properties as well~\cite{S2024D:CDMERDE}:
Attributes can be \emph{simple}, which means that they have atomic values that cannot be subdivided any further, like phone numbers.
Attributes can be \emph{composite}, which means that they consist of parts.
An address, for example, can be divided into country, province, city, etc.

It is not always immediately clear how an attribute could be modeled.
For example, our students have the attribute \glsreset{dateOfBirth}\emph{\pgls{dateOfBirth}}.
Technically, we have learned back in \cref{sec:factory:demand} that dates are atomic datatypes in \sql.
So naturally, we would model the \pgls{dateOfBirth} as an atomic attribute.
Of course, we could also model it as a composite attribute consisting of year, month, and day.
Then again, an address could also be represented as a single string of text instead of using a composite attribute.

The decision of how to model attributes probably depends on which data we need.
For example, if we store dates as, well, atomic dates, then it is extremely easy and fast to calculate the year of birth or the age of a person.
Storing years separately would be useless and just complicate and slow down the \db.
For an address, however, extracting the country from an address string could be tedious and error prone.
And we would probably need the country in almost any use case where an address is required.

Anyway, besides being either simple or composite, attributes can also be single-valued or multi-values.
Students can have multiple addresses, multiple phone numbers, but only a single \pgls{dateOfBirth}.%
%
\endhsection%
%
