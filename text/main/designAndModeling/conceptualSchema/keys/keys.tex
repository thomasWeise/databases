\hsection{Keys}%
\label{sec:conceptualKeys}%
%
In \cref{def:entity}, we stated that an entity can be distinguished from all other entities in the world.
This means that it must be unique.
Entities are characterized entirely and only by their attributes.
They do not have an property beyond their attribute values.
The only way it can be unique is because of its attributes.

The attributes that can serve as unique identifiers are called \emph{keys}~\cite{S2024D:CDMERDE}.%
%
\begin{definition}[Super Key]%
\label{def:superKey}%
A \emph{super key} is an attribute or set of attributes of an entity type that uniquely identifies an entity in an entity set~\cite{S2024D:CDMERDE,G2011EW2ITDS:CMUTERM}.%
\end{definition}%
%
A student entity, for example, can uniquely be identified by the student~ID.
It can also be uniquely identified by the combination of the address and mobile phone number(s).
Or just by the mobile phone number {\dots} if these were not optional.
Or maybe by an email address.
Or by the government-issued ID {\dots} if these were not optional.
Or we could try using the name, address, and \pgls{dateOfBirth}.%
%
\begin{definition}[(Candidate) Key]%
\label{def:key}%
A \emph{key}~(or \emph{candidate key}) of an entity type is a minimal super key, i.e., a super key that either consists of a single attribute or that would lose its unique property is a single attribute was removed from it~\cite{S2024D:CDMERDE,G2011EW2ITDS:CMUTERM,SS2005EIDDDFDB:SDLDUTRDM}.%
\end{definition}%
%
This does not necessarily mean that all candidate keys have the same number of attributes.
For example, when identifying a person, one candidate key could be the government issued~ID number.
Another possible candidate key could be a combination of the name, place of birth, and \pgls{dateOfBirth}.
Both would be minimal in the sense of the above definition.
The government-issued ID, however, would consist of a single attribute whereas the other choice consists of three.
The combination of all four attributes would be a super key, but not a candidate key.
Another super key would be the combination of name, gender, place of birth, and \pgls{dateOfBirth}.
It would not be a candidate key, because we could remove the attribute \inQuotes{gender} without impairing the uniqueness property.

If we ignore the issue of some attributes being optional for now, then we have at least three different candidate keys for students:
the student~ID, the government-issued ID, and the mobile phone number.%
%
\begin{definition}[Primary Key]%
\label{def:primaryKey}%
The \emph{primary key} of an entity type is a \emph{candidate key} that is used as \emph{the} identifying attribute or group of attributes of an entity when modeling relationships between different entity types.%
\end{definition}%
%
\begin{definition}[Prime Attribute]%
An attribute is referred to as \emph{prime} if it is part of the primary key.%
\end{definition}%
%
An entity cannot be referenced in one place using some attributes and in another using other attributes.
That would lead to all kinds of problems and inconsistencies.
Thus, there can only be one such primary key for each entity type.

Later in our \db\ design, we will also model how students enroll into modules.
Thus, we then also will have another entity type for modules~(or module instantiations in specific semesters).
It will be necessary to establish relationships between student and module-instance entities.
We already did something like this back in our factory example.
You do remember the \sqlil{PRIMARY KEY} and \sqlil{REFERENCES} keywords from back then.
Of course, these are technology-specific things which we do not care about at the conceptual level.
However, even at this level, we still must decide about which attributes we will \emph{actually} use to identify entities.

We will need one primary key that is used to uniquely identify students.
Which of the candidate keys makes the most sense?

How about mobile phone numbers?
Phone numbers may change.
Primary keys must never change.
Also, a student may have multiple phone numbers.
This would feel awkward to use a primary key.
Actually, some students might not have a mobile phone number.
This may be rare, but it could happen, as we already discussed before.
Primary keys must never be \sqlilIdx{NULL}.
So we rule out phone numbers as primary keys.

This is a very similar reason to \emph{not} use the government-issued ID number~(中国公民身份号码\cite{GB116431999CIN}) candidate key as primary key:
Foreign exchange students, so-called 留学生, do not have IDs issued by the Chinese government.
As said, primary keys should never be \sqlilIdx{NULL}, so we rule out government-issued IDs as well.

Another candidate key could be a combination of the name, place of birth, and \pgls{dateOfBirth}.
Another criterion for primary keys is that they should be reasonably small.
For example, composite attributes or attributes that consist of longer texts are not very suitable.
If we store relationships between entities, then we do this by storing their primary keys.
Recall, for example, our \sqlil{demand} table from back in \cref{sec:factory:demand}.
This means that primary keys are not just stored as part of their original entities, but also as part of all of the relations they are involved in.
While this is a technological aspect that does not really belong into the conceptual model design stage, it is something that we should keep in mind:
Huge keys are bad.
And the combination of name, place of birth, and \pgls{dateOfBirth} would be fairly impractical to store in multiple locations.
So we rule it out, too.

We would naturally prefer the student~ID that the university itself issues.
The reason is as follows:
A student joins a curriculum, maybe studies in the Bachelor Program \inQuotes{Computer Science and Technology} at our university.
For this process, they receive a student~ID.
This student~ID does not just represent them as a person, but it represents them \inQuotes{as a person in the function \inQuotes{BSc student of Computer Science and Technology.}}
Later, after graduation, they may join a Master's program and get a new, different student~ID.

This realization makes us feel a bit anxious about or concept of modeling students{\dots}
We may have two student entities referring to the same person, but at different stages of their educational process.
Maybe we should model students differently?
We will see.
For now, we accept this and use the student ID as primary key.
From the available candidate keys, it is the best option.

\begin{figure}%
%
\centering%
\includegraphics[scale=0.6]{\currentDir/erdStudent5}%
\caption{A new version of the \emph{Student} \pgls{ERD} from \cref{fig:erdStudent4}, this time with \emph{Student-ID} marked as primary key.}%
\label{fig:erdStudent5}%
\end{figure}%
%
We update our \pgls{ERD} from \cref{fig:erdStudent4}.
In the new \cref{fig:erdStudent5}, the attribute \emph{Student-ID} is marked as primary key.
This is done by underlining the attribute name~\cite{G2011EW2ITDS:CMUTERM}.

We learned a few things about primary keys.
Let's re-iterate them:%
%
\bestPractice{primaryKeys}{%
Primary keys should:%
\begin{enumerate}%
\item be unique for each entity~(obviously),%
\item be immutable over the lifetime of an entity,%
\item not be optional, i.e., they should never be allowed to be \sqlilIdx{NULL},%
\item not be derived attributes,%
\item always be single-valued attributes, i.e., not be multivalued attributes,%
\item consist of single attributes, i.e., not be based on candidate keys consisting of multiple attributes,%
\item be simple attributes, i.e., not composed attributes,%
\item be small in terms of the expected required storage size~(see also \cref{bp:surrogatePrimaryKey}).%
\end{enumerate}%
}%

Sometimes, it can happen that we end up with entity types where \emph{no} suitable primary key exists.
Maybe all the attributes that form candidate keys are just too long.
In such a case, we can use a technique we already learned in back in \dref{sec:factory:table:product}:%
%
\begin{definition}[Surrogate Key]%
\label{def:surrogateKey}%
When no suitable candidate key for an entity type exists, an artificial key, such as an auto-incremented integer value, can be used as \emph{surrogate key}.%
\end{definition}%
%
Keys are a very important component when we model the real world.
Each object in the real world must be unique in some aspect.
Each object must be identifiable by some unique properties.
The objects are represented as entities.
Their properties are represented as attributes.
Those attributes that we can use to identify them form the candidate keys.
Those attributes that we \emph{actually} use to identify them form the primary key.
Primary keys must be unique and small.%
\FloatBarrier%
\endhsection%
%
