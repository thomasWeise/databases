\hsection{Requirements Analysis}%
%
The requirements analysis is one of the most important steps of the \db\ development lifecycle.
It is this point where we gain the understanding of the project.
On one hand, we need to build a clear understanding of the purpose, goals, and limits of the project.
On the other hand, we also need to learn about the organizational data and processes that should be embodied by our \db\ and the applications on top of it.
Indeed, studies show that for many companies, more than half of the problems of systems~\cite{Z2003RDARS} and costs of software development~\cite{IC2009BAB2TPTS} are based on poor requirements definition.
Inefficient requirements management is considered a top-cause for project failure~\cite{EDN2005RUIFACI}.
Good requirements engineering can increase the developer productivity and lead to improved project planning~\cite{DCVP2005READSDFFAC}.
Fixing errors in the requirement cost 10 to 200 times as much to fix once the application is deployed compared to discovering during the requirements analysis~\cite{BP1988UACSC,M2001FTEAOOP,RGJ2023EASARBFSAR}.%
%
\hsection{Types of Requirements}
Requirements can roughly be divided into business requirements, functional requirements, non-functional requirements, and constraints~\cite{I2018SAH}.

First, the business requirements are the high-level goals, objectives, and sought outcomes of the project.
An organization usually has some overall initiative or plan to improve some of its metrics.
The project is initiated to support this initiative.
These requirements are more general and abstract.

Second, the functional requirements are more concrete and define what the system should do.
They concern the features and the behavior that the system must offer to fulfill the business requirements.
They may be defined as the input given to the system, the expected operation to be performed by the system on that input, and the expected output to be produced.

Third, the non-functional requirements define how the system should perform.
They define the service quality that the system should offer, the performance, usability, scalability, reliability, etc.
This includes the computational environment in which it must be possible to execute the system.

Fourth and finally, the constraints are the factors that limit which solutions are viable.
They define the boundaries in which we operate.
While requirements define properties that our system should have, the constraints rule out methods to get there.
They can appear in form of budget or time limits.
They may appear in the form of \inQuotes{your system must work with version~XYZ of software~ABC.}%
\endhsection%
%
\hsection{Requirements Gathering}%
%
Several different methods exist that can be used to collect requirements~\cite{Z2003RDARS}.

The first method, interviewing stakeholders to gather information about the system that we will design, is considered one of the most efficient requirements gathering techniques~\cite{DTHJM2006EORETERDFASR,Z2003RDARS}.
For each interview, careful preparation is needed, which includes making an appropriate appointment.
While stakeholders may spontaneously share information on some topics, they may not discuss others unless explicitly prompted with questions~\cite{BJF2014WSWOWNSATAESOTIIREEI}.
Each interview meeting should therefore have a proper agenda, pre-prepared questions, and follow a checklist~\cite{WSEG2016ETIOAKOSTCORQQ}.
If possible and if the interview partner(s) consent, then the meeting should be recorded.
The recordings and notes should be evaluated within two days after the interview.
The interviews should neutral, not push the interviewee into any direction.
The goal is to collect diverse views on the project.

A second, more standardized method to collect information are questionnaires~\cite{TAE2008ISRTDQTSTDQSSAE,Z2003RDARS}.
This way, we can collect much data from many stakeholders within a brief time.
Questionnaires are easy to evaluate and process, but designing them properly is important.
Different types of stakeholders will use different parts of our system.
Therefore, it may be necessary to design several different questionnaires, one for each group of future users.
The questions should be clear an unambiguous.
For some aspects, multiple-choice questions or range-based ratings are good, while others may require open-ended questions where the users can fill in their opinions.

The third method is to directly observe users doing their work~\cite{SGGM2007TMSPATFISRDWS,Z2003RDARS}.
Stakeholders may not always be able to accurately describe their function and how they fulfill it.
Observing them performing their day-to-day processes can thus provide helpful additional information.
It is also possible to record such real-world examples instead of personally observing them~\cite{HPW1998REAVWRWS}.
Of course, under observation, people may behave somewhat differently from normal, so such information is to be taken with a grain of salt.

Fourth, we can also analyze both organization-internal and external regulations, as well as published procedures, processes, and other documents~\cite{RGJ2023EASARBFSAR,Z2003RDARS}.
Usually, an organization will have their own written regulations, announcements, and otherwise standardized procedures.
Additionally, there are regulations and laws imposed upon an organization and the processes within.
Gathering such information can be crucial and complements our understanding of what the system is supposed to do, and why current systems work the way they do.
Sometimes, the official documentation of organization-internal methods and the practical realization of processes may differ, though.

A fifth method are group meetings and workshops~\cite{Z2003RDARS}.
Here, under the guidance of a session leader, stakeholders at all levels meet, from management to end user, from system analyst to data entry personnel.
The group then jointly discusses the current situation and the planned system, which can be a highly efficient way to gather requirements.
Different variants of this method exist since the 1970s, under names such as \glsreset{JAD}\pgls{JAD}~\cite{CWG1993PAJADATC,M1996RTWSS} and \glsreset{participatoryDesign}\pgls{participatoryDesign}~\cite{CWG1993PAJADATC,FMRSW1989OOSAATSDASD}.

Several of these techniques can be combined, often with other approaches such as brainstorming sessions, surveys, reverse engineering of existing systems, or prototyping~\cite{I2018SAH,Z2003RDARS}.
In the context of \dbs, it is particularly important to properly define the data structures and entities when analysing the requirements and, later, when developing the conceptual schema~\cite{M1987DFADSMFDRACS}.%
%
\endhsection%
%
\hsection{Requirements Specification Document}%
After the requirements have been collected, they are stored in a formal specification document, the so-called \glsreset{SRS}\pgls{SRS} document~\cite{S2010DSRSAR,W2004ASOTMFMTQOTRSD}.
The \pgls{SRS} is the most important document in the software development process.
The document structure should follow the IEEE~830\nobreakdashes-1998~\cite{IEEE1998IRPFSRS} standard or the newer ISO/IEC/IEEE~29148\nobreakdashes-2018~\cite{ISOIECIEEE2018SASELCPRE} standard.%
\endhsection%
%
\endhsection%
%
