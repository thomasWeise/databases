%
\hsection{Getting Started}%
%
The goal of this course is to introduce databases.
However, this is a practical course.
So we will work on practical examples and create \inQuotes{real} \pglspl{db} on real \pglspl{dbms}.
This means that we will look into several interesting topics:%
%
\begin{itemize}%
%
\item We will work with an actual \pgls{dbms}. %
So you need to install an actual \pgls{DBMS}. %
We choose \postgresql~\cite{TA2024DDAMWPAM,FP2023LP,OH2017PUAR,B2024PELUYDW}.%
%
\item \pglspl{DBMS} are usually just servers to which you can send \sql\ commands via a \pgls{client} \pgls{terminal}. %
However, there exist also nicer tools that offer rich \pglspl{GUI} that allow you to design forms and reports. %
We need such a tool as well. %
We choose \libreofficeBase~\cite{FNFHWSKLSSGLFRSRPLJG2022BG7R1BOL7C,S2022L7PFEUU}.%
%
\item The \pgls{DBMS} is usually just the backend of a rich landscape of tools in an enterprise or organization. %
Rarely will one work only with a \pgls{DBMS}. %
Instead, there may be several applications that connect to \pglspl{db} through unified \pglspl{API}. %
Since we will also take a look at how that works, we need to install a programming language and corresponding \inQuotes{\pgls{DBMS}-access library.} %
We chose the \python\ programming language~\cite{K2018EIPFEUU,A2002PC,H2023ABGTP3P,LH2015DSAAWP}, because we also provide a free book on it in~\cite{programmingWithPython}. %
As library to connect to the \postgresql\ \pgls{DBMS}, we pick \psycopg~\cite{VDGE2010P}.%
%
\end{itemize}%
%
As you can see, we only consider free open source software.
Learning about \pglspl{db} with this book should be free.

Before we get into the necessary installation and setup steps for the software that we need to really learn about \pglspl{db}, we face a small problem:
Today, devices with many different \pgls{OS} are available.
For each \pgls{OS}, the installation steps and software availability may be different, so I cannot possibly cover them all.
Personally, I strongly recommend using \linux~\cite{T1999TLE,B2022ELATCL,H2022LML} for programming, work, and research.
If you are a student of computer science or any related field, then it is my personal opinion that you should get familiar with this operating system.
Maybe you could start with the very easy-to-use \ubuntu\ \linux~\cite{CN2020ULB,H2020ULU2E}.
Either way, in the following, I will try to provide examples and instructions for both \ubuntu\ and the commercial Microsoft \windows~\cite{B2023W1IO} \pgls{OS}.%
%
\hinput{installingPostgres}{installingPostgres.tex}%
\hinput{installingLibreOffice}{installingLibreOffice.tex}%
\hinput{installingPythonEtc}{installingPythonEtc.tex}%
%
\endhsection%
%
