%
\hsection{Getting Started}%
%
The goal of this course is to introduce \pglspl{db}.
However, this is a practical course.
So we will work on practical examples and create~(more or less) \inQuotes{realistic} \pglspl{db} on real \pglspl{dbms}.
This means that we will look into several interesting topics.
And for most topics, we will try to get practical hands-on experience using a suitable software tool.%
%
\begin{itemize}%
%
\item We will work with an actual \pgls{dbms}.
So you need to install an actual \pgls{dbms}.
We choose \postgresql~\cite{TA2024DDAMWPAM,FP2023LP,OH2017PUAR,B2024PELUYDW}.%
%
\item \pglspl{dbms} are usually just \pglspl{server} to which you can send \sql\ commands via a \pgls{client} \pgls{terminal}.
However, there exist also nicer tools offering rich \pglspl{GUI} that allow you to design forms and reports.
Forms are structured graphical masks for data entry.
Reports are documents that are automatically filled with data from a \db.
We will try using such a tool as well.
We choose \libreofficeBase~\cite{FNFHWSKLSSGLFRSRPLJG2022BG7R1BOL7C,S2022L7PFEUU}.%
%
\item The \pgls{dbms} is usually just the backend of a landscape of tools in an enterprise or organization.
Rarely will users work only with or directly on a \pgls{dbms}.
Instead, there may be several applications that connect to \pglspl{db} through unified \pglspl{API}.
Since we will also take a look at how that works, we need to install a programming language and corresponding \inQuotes{\pgls{dbms}-access library.}
We chose the \python\ programming language~\cite{K2018EIPFEUU,A2002PC,H2023ABGTP3P,LH2015DSAAWP}, because we also provide a free book on it in~\cite{programmingWithPython}.
As library to connect to the \postgresql\ \pgls{dbms}, we pick \psycopg~\cite{VDGE2010P}.%
%
\item An important step of the \pglspl{db} design process is to create an abstract conceptual model of the problem domain.
This involves drawing graphical diagrams, so-called \glsreset{ERD}\pglspl{ERD}, describing the real-world entities and their relationships that should be modeled in the \db.
The conceptual schema should be independent from any specific \dbms\ technology.
As tool for drawing such diagrams, we will use \yEd~\cite{SG2015MDAWY,Y2011YGEM}.%
%
\item Conceptual models need to eventually be mapped to logical schemas that are based on certain data models and \db\ technologies.
The logical model is how users and applications see the data.
They may be implemented directly using a language like~\sql.
Interestingly, technology-specific conceptual models can also be designed using visual tools very similar to \pglspl{ERD}, which, this time, are bound to specific technologies and can be translated to commands for a \dbms.
In my opinion, the best open source and free tool for drawing logical models for the \postgresql\ \dbms\ is \pgmodeler~\cite{AES2006PPDM}. %
So we will use it here as well.%
%
\end{itemize}%
%
\begin{figure}%
\centering%
\begin{noglslink}%
%
\floatSep%
%
\subfloat[][%
The \href{https://www.python.org}{\python\ programming language} logo is under the copyright of its owners.%
]{\parbox[t]{0.45\linewidth}{\centering\includegraphics[width=0.9999\linewidth]{\currentDir/pythonLogo}}}%
%
\floatSep%
%
\subfloat[][%
The logo of the \href{https://www.psycopg.org}{\psycopg\ module}, the \python\ library for accessing \postgresql. %
Copyright \textcopyright~Gabriella Albano and the Psycopg team.%
]{\parbox[t]{0.45\linewidth}{\centering\includegraphics[width=0.45\linewidth]{\currentDir/psycopgLogo}}}%
%
\floatSep\floatRowSep\floatSep%
%
\subfloat[][%
The logo of the \href{https://www.yworks.com/products/yed}{\yEd\ graph editor}. %
The \yEd~logo is protected by copyright. %
\yEd~is a registered trademark of \href{https://www.yworks.com}{yWorks~GmbH}. %
Unauthorized use, reproduction, or distribution is strictly prohibited.%
]{\parbox[t]{0.45\linewidth}{\centering\includegraphics[width=0.5\linewidth]{\currentDir/yEdLogo}}}%
%
\floatSep%
%
\subfloat[][%
The \href{https://pgmodeler.io}{\pgmodeler\ logo} is under the copyright of Raphael Araújo~e~Silva.%
]{\parbox[t]{0.45\linewidth}{\centering\includegraphics[width=0.5\linewidth]{\currentDir/pgmodelerLogo}}}%
%
\floatSep%
%
\caption{The logos of several of the very nice and free tools that we are using.}%
\end{noglslink}
\end{figure}%
%
\bestPractice{manyTools}{%
A computer science professional is able and always keen to learn new tools. %
A computer science professional should know dozens if not hundreds of different software tools for different tasks. %
A software engineer is a craftsperson and their knowledge of software is their tool belt.%
}%
%
We will use some specific \dbms, some specific visualization tools, some specific programming library to access the specific \dbms, and so on.
One may wonder whether this is a good idea.
If you would ever work professionally with \pglspl{db}, most likely you will use entirely different software.
Will the knowledge we learn be useless then?

No.
First of all, we will discuss the fundamentals of \pglspl{db} and just use the concrete technologies as examples of how they play out in practice.
So the knowledge about the fundamental concepts stays valuable, regardless of which tool you use.

Second, theoretical knowledge is not very useful in a professional setting if you cannot actually apply it.
It does not help you if you have tried writing \sql\ code on paper but never actually executed it on a real \dbms.
You do not know how to connect to a \dbms, how to input commands, how to understand error message.
You would also never have searched for specific commands and documentation in the internet.
You maybe would never have dealt with installing software and get it to run properly.
This means that any practical application would still require you to learn lots of things and, most likely, under time pressure.
However, if you \emph{know} how to install \postgresql, how to connect to it using the \psql\ client, if you actually have executed \sql\ commands, if you made mistakes and learned how to figure out how to fix them {\dots} then most likely you can relatively quickly learn how to do that for \mysql, \mariadb, or any other \dbms.

If you have \emph{seen} \pglspl{ERD} and can read them, this does not mean that you can \emph{efficiently} draw some if you are asked to do so.
However, if you have drawn some using \yEd\ or \pgmodeler, then at least you know that such tools exists and how they are used.
You can probably either adapt to a new tool or find and install them if need be.
But you would probably not start drawing such diagrams using a tool not suitable for that, like, e.g., a general graphics program such as~\inkscape.

All the struggle of installing software, using command line arguments, connecting \pglspl{client} to \pglspl{server}, failing, and finding solutions will help you when you need to do similar things for entirely other software.
Usage paradigms and even fixes for errors are often similar over different tools, so the more tools you use, the faster you become learning how to use other tools.
Therefore, I strongly advocate not just learning the fundamentals of \pglspl{db}, but to actually try them out, exercise them, use tools.

Finally, all the tools that we consider here are free and, ideally, open source, software.
We do not use tools that cost money.
Learning about \pglspl{db} with this book should be free.%
%
\noviceHint{%
In this part of the book, we have collected installation instructions for all the tools that we use in those book.
{\color{red}{\textbf{You do not need to install all of them right away.}}}
Just be aware that we provide installation instructions for the tools that we need here.
When we eventually need the specific software tools at some time later on, we will refer back to this part.%
}%
%
Either way, before we get into the necessary installation and setup steps for the software that we eventually need to really learn about \pglspl{db}, we face a small problem:
Today, devices with many different \pgls{OS} are available.
For each \pgls{OS}, the installation steps and software availability may be different, so I cannot possibly cover them all.
Personally, I strongly recommend using \linux~\cite{T1999TLE,B2022ELATCL,H2022LML} for programming, work, and research.
If you are a student of computer science or any related field, then it is my personal opinion that you should get familiar with this operating system.
%
\bestPractice{knowLinux}{%
Any professional computer scientist, software developer, software architect, \pgls{dba}, or system administrator should be familiar with the \linux\ \pgls{OS}.%
}%
%
Maybe you could start with the very easy-to-use \ubuntu\ \linux~\cite{CN2020ULB,H2020ULU2E}.
If you are a \microsoftWindows\cite{B2023W1IO} user, maybe you could install \ubuntu\ in a virtual machine.
Either way, I strongly recommend learning and using \linux.

It should also be clear that the instructions provided here will eventually be outdated.
I am not sure whether I will be able to keep updating them in the future.
However, even a few years down the road, they should still provide some basic guidance.
In the following, I will try to provide examples and instructions for both \ubuntu\ \linux~\cite{CN2020ULB,H2020ULU2E} and the commercial \microsoftWindows~\cite{B2023W1IO} \pgls{OS}.%
%
\hinput{examples}{examples.tex}%
\hinput{installingPostgres}{installingPostgres.tex}%
\hinput{installingLibreOffice}{installingLibreOffice.tex}%
\hinput{installingPythonEtc}{installingPythonEtc.tex}%
\hinput{installingYed}{installingYed.tex}%
\hinput{installingPgModeler}{installingPgModeler.tex}%
%
\endhsection%
%
