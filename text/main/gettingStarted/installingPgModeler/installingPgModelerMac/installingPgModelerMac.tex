%
\hsection{Installing PgModeler under MacOS}%
\label{sec:installingPgModelerMacOS}%
%
Installing the \pgmodeler\ under \pgls{macOS} is similar to the installation under \microsoftWindows.
Since I do not have an Apple computer, this section needs to survive without screenshots.
The basic steps that you will have to take are as follows:%
%
\begin{enumerate}%
%
\item Go to \url{https://www.macports.org} and then to the register \emph{\inQuotes{Installing \pgls{macPorts}}} -- or directly visit
\url{https://www.macports.org/install.php}.%
%
\item Download the \pgls{macPorts} version suitable for your version of \pgls{macOS}. %
For example, if you have \inQuotes{\pgls{macOS} Ventura}, then you would download the installer for \inQuotes{\pgls{macOS} Ventura}%
%
\item Install the downloaded version of \pgls{macPorts}.%
%
\item Update the package list of the \pgls{macPorts} system.%
\begin{enumerate}%
%
\item Open a \emph{new} \pgls{terminal} window.%
%
\item Write \bashil{sudo port -v selfupdate} and press~\keys{\enter}. %
It will ask you for your (\pgls{sudo}) password, i.e., the (administrator) password for your computer. %
Enter it an hit~\keys{\return}.%
%
\item After the command completes, close the \pgls{terminal}.%
\end{enumerate}%
%
\item We also need \pgls{xcode} as a dependency. %
So if you do not yet have it installed\dots%
\begin{enumerate}%
\item Open a \emph{new} \pgls{terminal} window.%
%
\item Type in \bashil{xcode-select --install} and press~\keys{\enter}. %
If you need to provide your password again, do so.%
%
\item After the installation completes, close the \pgls{terminal}.%
\end{enumerate}%
%
\item Finally, we should be able to install the \pgmodeler\ package.%
\begin{enumerate}%
\item Open a \emph{new} \pgls{terminal} window.%
%
\item Enter \bashil{sudo port install pgmodeler} and press~\keys{\enter}. %
It will ask you for your (\pgls{sudo}) password, i.e., the (administrator) password for your computer. %
Enter it an hit~\keys{\return}.%
%
\item After the installation completes, close the \pgls{terminal}.%
\end{enumerate}%
%
\item Open a \emph{new} \pgls{terminal} window.%
%
\item You should now be able to execute \pgmodeler\ using the command \bashil{pgmodeler} + \keys{\enter}.%
%
\end{enumerate}%
%
The above follows a similar pattern compared as in the \microsoftWindows\ situation.
We first need some abstraction layer that makes \linux\ software available on this \pgls{OS}.
What \pgls{MSYS2} offers for \microsoftWindows, \pgls{macPorts} offers for \pgls{macOS}~\cite{RRJ2008MFUG}.
This time, we also need an additional dependency, namely \pgls{xcode}.
If it is not yet installed, we install it.
After that, \pgmodeler\ can be installed.
And after that, it can be used.%
%
\endhsection%
%
