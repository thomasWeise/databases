%
\hsection{Installing PostgreSQL}%
\label{sec:installingPostgres}%
%
For our course, we will use the open source \pgls{dbms} \postgresql~\cite{TA2024DDAMWPAM,FP2023LP,OH2017PUAR,B2024PELUYDW} as the actual system to experiment with \pglspl{db}.
Here, we briefly discuss how to install it on your machine.
We will then use \postgresql\ in \cref{sec:simpleExampleFactory}.

\postgresql\ follows the \pgls{clientServerArchitecture}.
It provides the \pgls{dbms} implemented as \pgls{server} program.
It manages the \pglspl{db} and stores and provides the data.
Other programs and the users can connect to it to access the \pglspl{db}.
\postgresql\ also offers a \pgls{client} program, \psql, with which human users can communicate with the \pgls{dbms} \pgls{server} in a normal \pgls{terminal}.
We want to install both pieces of software.
You can find more information about this process at the \postgresql\ download page \url{https://www.postgresql.org/download}.
Here, we provide step-by-step guides for \ubuntu\ \linux\ and \microsoftWindows.%
%
\hinput{installingPostgresUbuntu}{installingPostgresUbuntu.tex}%
\hinput{installingPostgresWindows}{installingPostgresWindows.tex}%
%
\endhsection%
%
